\chapter{Referencial Teórico}\label{chp:LABEL_CHP_REF_TEO}

Na sociedade contemporânea, é notória a influência que exerce a tecnologia. Se há poucas décadas o computador se introduzia lentamente na rotina dos cidadãos, hoje está presente não apenas nas mesas de trabalho, na forma de \textit{desktops} e \textit{notebooks}, mas também nos bolsos, na forma de celulares, e, mais recentemente, em várias partes do corpo, na forma de relógios, óculos e pulseiras inteligentes, dentre outros \textit{wearables}.

Diante de um cenário de constante mudança, é natural que as instituições mais tradicionais tenham dificuldades em acompanhar a rápida evolução das NTIC (Novas Tecnologias de Informação e Comunicação). Neste contexto, a situação da escola é delicada pois ela lida diretamente com jovens que muitas vezes possuem amplo acesso a tecnologias de ponta logo nos primeiros anos de vida, e demonstram ser donos de uma perícia que está muito além das habilidades de seus professores.

Por outro lado, a escola também recebe jovens que possuem pouco ou nenhum acesso à tecnologia, e que não dispõem de conhecimentos básicos acerca dela. Sobre isso, afirma Mattos \cite{art:REF_ART_MATTOS} que, em 2008, apenas 21\% da população brasileira e apenas 26,7\% da população fluminense utilizava a Internet. A desigualdade no acesso à tecnologia, um traço tão marcante da sociedade contemporânea, nunca esteve tão bem representada na sala de aula.

\section{A Importância da Inclusão Digital}\label{sec:LABEL_CHP_REF_TEO_SEC_ID}

Neste contexto, a inclusão digital promovida a partir da escola assume importância fundamental para reduzir a desigualdade social do nosso país. A respeito do Proinfo, programa de inclusão digital do Governo Federal, afirmam Carvalho e Monteiro \cite{art:REF_ART_CARVALHO_MONTEIRO} que "a preocupação oficial no Brasil em relação à introdução da informática nas redes de ensino, bem como sobre seu acesso, usos e possibilidades na escola alia-se a discussões sobre a necessidade de equalizar as oportunidades de inclusão digital para grupos sociais excluídos".

A inclusão digital, que antes era considerada ferramenta de redução da desigualdade social, hoje se configura como ferramenta essencial para o não aprofundamento desta desigualdade, uma vez que "na realidade brasileira, as pessoas participantes de diversos contextos sociais estão tendo um crescente acesso a computadores enquanto ferramenta de trabalho" \cite{art:REF_ART_CARVALHO_MONTEIRO}.

A habilidade no uso das NTIC vem sendo colocada pelos empregadores como exigência mínima para a contratação para as suas vagas de emprego, ainda que em cargos que exijam pouca formação acadêmica e que possuam baixa remuneração. Se os grupos sociais excluídos permanecerem sem acesso às NTIC, estes terão cada vez menos competitividade no mercado de trabalho, o que contribuirá para aprofundar ainda mais as desigualdades sociais.

\section{A Importância da Informática Educativa}\label{sec:LABEL_CHP_REF_TEO_SEC_IE}

A inclusão digital, apesar de importante, não deve ser o único aspecto da informática nas escolas, que têm como missão a realização dos objetivos pedagógicos e educacionais. O computador é uma ferramenta poderosa, que em muito pode contribuir para a pedagogia. "Entretanto, constata-se em algumas realidades escolares, a ausência de uma visão mais ampliada da utilização do computador enquanto um mediador na construção de conhecimentos" \cite{art:REF_ART_CARVALHO_MONTEIRO}.

Neste contexto, afirma Dutra \cite{art:REF_DISS_DUTRA} que "o uso das NTIC no ensino constitui em um dos aspectos considerados importantes para a modernização da escola e da educação. A essas tecnologias é depositada a possibilidade de transformar o ensino, de dar mais autonomia ao aluno, alterando a relação professor-aluno e aponta para mudanças na estrutura e funcionamento da escola". Entretanto, "enquanto a sociedade muda e experimenta desafios mais complexos, a educação formal continua, de maneira geral, organizada de modo previsível, repetitivo, burocrático, pouco atraente" \cite{art:REF_LIVRO_MORAN}.

Fica evidente a necessidade de a escola introduzir na sua rotina as NTIC, uma vez que elas podem ser um agente catalisador do processo de aprendizado, contribuindo efetivamente para este, além de atuarem como fator motivacional para os alunos, que se sentem felizes e demonstram maior engajamento ao projeto pedagógico quando este envolve o uso da tecnologia.

Apesar de concordar com tais afirmações, Odorico \cite{art:REF_ART_ODORICO} realiza uma ressalva: "Atualmente, a informática cresce em uso e relevância em todo o mundo possibilitando trazer inúmeros desenvolvimentos de atividades diferenciadas para a sala de aula. As atividades utilizando recursos computacionais podem contribuir de forma efetiva para o processo ensino aprendizagem, desde que haja contribuição dos programas governamentais, apoio técnico, capacitação e planejamento dos educadores".

Eis que surge um dos desafios da Informática Educativa (IE): como criar e executar, em uma escola, um projeto pedagógico que utilize de maneira significativa as NTIC e atenda tanto a alunos que possuem alto grau de acesso à tecnologia (e apresentam admiráveis habilidades com ela) quanto a alunos que possuem pouco ou nenhum grau de acesso? Um ponto importante e ainda mais desafiador: como elaborá-lo sem introduzir segmentações ou discriminações entre esses grupos de alunos?

\section{O Laboratório de Informática Educativa}\label{sec:LABEL_CHP_REF_TEO_SEC_LIE}

O primeiro passo, evidentemente, surge de modo natural: é necessário que a escola possua equipamentos disponibilizados em uma sala adequada para que seja possível executar este projeto pedagógico. Em outras palavras, é necessário que a escola possua um Laboratório de Informática Educativa (LIE).

Entretanto, apenas a construção de um LIE não é o suficiente para uma implementação eficiente da IE. Sobre isso, afirma Odorico \cite{art:REF_ART_ODORICO} que "muitas instituições que inseriram Laboratórios de Informática em seu meio não conseguiram realizar as modificações esperadas, sendo estes espaços considerados na maioria dos casos enfeites".

O que se observa é que existe a ideia, construída pela mídia e utilizada como \textit{marketing} por algumas instituições de ensino, "de que com laboratórios instalados nas escolas teremos automaticamente cursos melhores e resolvidos os nossos centenários problemas educacionais (...). Como em outras épocas, há uma expectativa de que as novas tecnologias nos trarão soluções rápidas para mudar a educação" \cite{art:REF_LIVRO_MORAN}.

É importante ter em mente que a tecnologia, por si, não modifica projetos pedagógicos e não oferece melhores resultados acadêmicos. É necessário pensá-la como recurso didático, como ferramenta de ensino, e elaborar projetos pedagógicos levando em consideração as possibilidades e limitações da tecnologia, de modo que se possa aproveitá-la em todo o seu potencial. Conforme colocado por Moran \textit{et al} em seu livro \cite{art:REF_LIVRO_MORAN}, "não são os recursos que definem a aprendizagem, são as pessoas, o projeto pedagógico, as interações, a gestão".

O Laboratório de Informática Educativa deve ser considerado como recurso didático da escola, e não apenas como uma sala de informática, onde os alunos (nem sempre) podem entrar. É necessário que o LIE seja levado em conta no momento da elaboração dos projetos pedagógicos, seja da escola como um todo, onde o laboratório pode atuar como ferramenta multidisciplinar, unificando objetivos de componentes curriculares diferentes; seja das disciplinas isoladamente, onde o professor pode (e deve) utilizar o laboratório como catalisador do processo de aprendizado, para chegar de maneira mais rápida e mais efetiva aos seus objetivos pedagógicos.

Ao tomarmos consciência de que este processo não é realizado nas escolas, é natural que busquemos as razões que levam a isto. A literatura aponta que as mais relevantes razões pelas quais a IE não é bem implementada mesmo com a presença do LIE são a resistência inicial dos professores da escola e dos gestores.

\subsection{O papel dos professores}\label{sec:LABEL_CHP_REF_TEO_SEC_RES_PROF}

A respeito da resistência inicial dos professores, relata Chagas \cite{art:REF_ART_CHAGAS} em sua investigação que "era transparente nos docentes, a resistência em querer integrar o LIE em suas atividades pedagógicas, pois já têm internalizado procedimentos didáticos-práticos sem o uso do computador. Vários são os termos usados pelos docentes em seus discursos quanto ao conhecimento dos recursos tecnológicos, tais como 'sou totalmente leigo em informática' ou 'sou leigobyte' ou 'analfabyte' ou 'informática é muito difícil'. Afirmações como estas vêm para reforçar a barreira inicial criada pelo professor diante do computador".

Entretanto, é necessário ter em mente que "o professor tem a responsabilidade de preparar o aluno para se tornar um cidadão ativo dentro da sociedade, apto a questionar, debater e romper paradigmas" \cite{art:REF_ART_OLIVEIRA}.

No contexto da sociedade contemporânea, onde a tecnologia exerce crescente influência em diversos setores, é razoável considerar que um cidadão ativo é um cidadão que sabe utilizar as NTIC, uma vez que o domínio delas é necessário para realizar atividades de cidadania, como declarar o Imposto de Renda. % [este assunto já foi abordado acima] É notório também que muitas empresas, não necessariamente grandes, têm inserido o domínio das NTIC nos requisitos de suas vagas de emprego, mesmo em funções que não necessitam de grande formação acadêmica, como caixa de supermercado ou secretário.

No âmbito tecnológico-escolar, o papel do professor é, portanto, preparar seus alunos para a utilização das tecnologias, familiarizando-os com elas e dando subsídios para que estes alunos, no futuro, sejam capazes de aprender por si mesmos a utilizá-las conforme sentirem necessidade.

Apesar de muitos alunos apresentarem pouca ou nenhuma familiaridade com a tecnologia, outros apresentam habilidades que em muito superam a de seus professores. Neste contexto, afirma Odorico \cite{art:REF_ART_ODORICO} que "muitos professores se sentem inseguros quando trabalham com seus alunos no laboratório de informática, pois sabem que estes dominam as máquinas e as ferramentas mais do que eles".

Se a barreira inicial criada pelo professor diante do computador tem como cerne a falta de habilidade deste com as tecnologias e há alunos que as dominam, então, tomando inspiração em Cora Coralina, que declarou que "feliz aquele que transfere o que sabe e aprende o que ensina", cria-se a oportunidade perfeita para que o professor aumente seus conhecimentos de tecnologia aprendendo com os próprios alunos.

Observa-se, portanto, que a principal dificuldade a ser transposta, da parte dos professores, para que eles cumpram com o seu papel é a resistência inicial que possuem.

\subsubsection{O docente do LIE}\label{sec:LABEL_CHP_REF_TEO_SEC_DOC_LIE}

Para amenizar este problema, é fundamental que esteja presente um ator conhecido como "docente do LIE". Este professor possui formação tanto técnica quanto pedagógica, e está capacitado para auxiliar os demais atores (gestores e outros professores) na elaboração e execução de atividades pedagógicas no contexto da Informática Educativa.

A importância da presença do docente do LIE para a implementação eficiente da IE é reforçada por Chagas \cite{art:REF_ART_CHAGAS}: "O desconhecimento técnico mínimo necessário para o uso das novas ferramentas de trabalho, dos softwares educativos, da Internet, de atividades no computador, são pontos fortes (sic) para a não implementação da IE (...). Neste momento, o apoio específico do professor do laboratório se torna indispensável para que se atinja as sonhadas mudanças da IE".

A despeito de tal importância, muitas vezes o docente do LIE não está presente nas escolas. Odorico \cite{art:REF_ART_ODORICO}, ao investigar sobre os laboratórios de duas escolas da rede estadual de ensino de Minas Gerais, constatou que "um fator que pesa muito na utilização dos laboratórios é a presença na escola de um profissional da área de informática, já que muitos professores sentem-se inseguros e apresentam dificuldades para utilizar o computador. Entretanto nenhuma das instituições analisadas possui tal profissional para auxiliar os professores nas matérias específicas".

\subsection{O papel dos gestores}\label{sec:LABEL_CHP_REF_TEO_SEC_RES_GEST}

É notório que os gestores das escolas possuem muitos encargos, que oneram seu trabalho e o tornam desgastante e burocrático. Sobre isso, afirma Groll \cite{art:REF_TCC_GROLL} que "a gestão escolar se confunde, frequentemente, com administração, e não raro com Administração Científica, onde a burocracia impera. Mas o gestor escolar tem, além das tarefas administrativas e burocráticas, seu papel de professor".

Por conta dos encargos das funções de gestão, por vezes os gestores não se mostram interessados em trabalhar melhor o LIE no contexto pedagógico, pois os diversos problemas estruturais se tornarão novos encargos nas suas funções. Entretanto, no contexto do seu papel de professor, "é necessário que os dirigentes escolares atentem para o significado do trabalho de informatização como meio para a realização dos objetivos educacionais de natureza pedagógica, razão última da existência da escola" \cite{art:REF_TCC_GROLL}.

O papel do gestor no processo de implementação da IE na escola é explicitado por Groll \cite{art:REF_TCC_GROLL}: "os gestores, enquanto líderes, devem ser os encaminhadores rumo à tomada de consciência de toda comunidade escolar sobre a nova realidade que vivenciamos e que, portanto, a escola deve ser um ambiente aberto, democrático, flexível e naturalmente adequado para a promoção da inclusão digital e favorecer (sic) a criatividade e a inovação mediante o diálogo entre os vários componentes da comunidade".

% Chagas relata a respeito de cursos de formação para docentes do LIE na página 4 de seu artigo: "Quanto a sua Formação: Cursos de Especialização em Informática Educativa, era um investimento da PMF/Proinfo, para estes professores. Mas, a própria Prefeitura exigia o cumprimento destes profissionais nas atividades na escola".
% Isso pode ou não ser incluído no texto, neste capítulo ou em outro.