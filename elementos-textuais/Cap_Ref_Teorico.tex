\chapter{Referencial Teórico}\label{chp:LABEL_CHP_REF_TEO}

Na sociedade contemporânea, é notória a influência que exerce a tecnologia. Se há poucas décadas o computador se introduzia lentamente na rotina dos cidadãos, hoje está presente não apenas nas mesas de trabalho, na forma de \textit{desktops} e \textit{notebooks}, mas também nos bolsos, na forma de celulares, e, mais recentemente, em várias partes do corpo, na forma de relógios, óculos e pulseiras inteligentes, dentre outros \textit{wearables}.

Diante de um cenário de constante mudança, é natural que as instituições mais tradicionais tenham dificuldades em acompanhar a rápida evolução das NTIC. Neste contexto, a situação da escola é delicada pois ela lida diretamente com jovens que muitas vezes possuem amplo acesso a tecnologias de ponta logo nos primeiros anos de vida, e demonstram ser donos de uma perícia que está muito além das habilidades de seus professores.

Por outro lado, a escola também recebe jovens que possuem pouco ou nenhum acesso à tecnologia, e que não dispõem de conhecimentos básicos acerca dela. A desigualdade no acesso à tecnologia, um traço tão marcante da sociedade contemporânea\todo{add fontes a isso?}, nunca esteve tão bem representada na sala de aula.

\section{A Importância da Informática Educativa}\label{sec:LABEL_CHP_REF_TEO_SEC_IE}

Neste contexto, fica evidente a necessidade de a escola introduzir na sua rotina as NTIC, uma vez que elas podem ser um agente catalisador do processo de aprendizado, contribuindo efetivamente para este, além de atuarem como fator motivacional para os alunos, que se sentem felizes e demonstram maior engajamento ao projeto pedagógico quando este envolve o uso das NTIC.

Neste contexto, afirma Dutra \cite{art:REF_DISS_DUTRA} que "o uso das NTIC no ensino constitui em um dos aspectos considerados importantes para a modernização da escola e da educação. A essas tecnologias é depositada a possibilidade de transformar o ensino, de dar mais autonomia ao aluno, alterando a relação professor-aluno e aponta para mudanças na estrutura e funcionamento da escola".

Apesar de concordar com tais afirmações, Odorico \cite{art:REF_ART_ODORICO} realiza uma ressalva: "Atualmente, a informática cresce em uso e relevância em todo o mundo possibilitando trazer inúmeros desenvolvimentos de atividades diferenciadas para a sala de aula. As atividades utilizando recursos computacionais podem contribuir de forma efetiva para o processo ensino aprendizagem, desde que haja contribuição dos programas governamentais, apoio técnico, capacitação e planejamento dos educadores".

Eis que surge um dos desafios da Informática Educativa (IE): como criar e executar, em uma escola, um projeto pedagógico unificado que utilize extensivamente as NTIC atendendo a grupos de alunos que possuem tão diferentes níveis de acesso às tecnologias sem introduzir segmentações ou discriminações?

\section{O Laboratório de Informática Educativa}\label{sec:LABEL_CHP_REF_TEO_SEC_LIE}

O primeiro passo, evidentemente, surge de modo natural: é necessário que a escola possua equipamentos disponibilizados em uma sala adequada para que seja possível executar este projeto pedagógico. Em outras palavras, é necessário que a escola possua um Laboratório de Informática Educativa (LIE).

Entretanto, apenas a construção de um LIE não é o suficiente para uma implementação eficiente da IE. Sobre isso, afirma Odorico \cite{art:REF_ART_ODORICO} que "muitas instituições que inseriram Laboratórios de Informática em seu meio não conseguiram realizar as modificações esperadas, sendo estes espaços considerados na maioria dos casos enfeites".

Diante deste cenário, a literatura aponta que há algumas razões pelas quais a IE não é bem implementada mesmo com a presença do LIE. As mais relevantes são a resistência inicial dos professores da escola e a resistência dos gestores.

\subsection{A resistência dos professores}\label{sec:LABEL_CHP_REF_TEO_SEC_RES_PROF}

A respeito da resistência inicial dos professores, relata Chagas \cite{art:REF_ART_CHAGAS} em sua investigação que "era transparente nos docentes, a resistência em querer integrar o LIE em suas atividades pedagógicas, pois já têm internalizado procedimentos didáticos-práticos sem o uso do computador. Vários são os termos usados pelos docentes em seus discursos quanto ao conhecimento dos recursos tecnológicos, tais como 'sou totalmente leigo em informática' ou 'sou leigobyte' ou 'analfabyte' ou 'informática é muito difícil'. Afirmações como estas vêm para reforçar a barreira inicial criada pelo professor diante do computador".

Ainda neste contexto, afirma Odorico \cite{art:REF_ART_ODORICO} que "muitos professores se sentem inseguros quando trabalham com seus alunos no laboratório de informática, pois sabem que estes dominam a máquinas e as ferramentas mais do que eles".

\subsubsection{O docente do LIE}\label{sec:LABEL_CHP_REF_TEO_SEC_DOC_LIE}

Para amenizar este problema, é fundamental que esteja presente um ator conhecido como "docente do LIE". Este professor, exclusivo do laboratório, possui formação tanto técnica quanto pedagógica, e está capacitado para auxiliar os demais atores (gestores e outros professores) na elaboração e execução de atividades pedagógicas no contexto da Informática Educativa.

A importância da presença do docente do LIE para a implementação eficiente da IE é reforçada por Chagas \cite{art:REF_ART_CHAGAS}: "O desconhecimento técnico mínimo necessário para o uso das novas ferramentas de trabalho, dos softwares educativos, da Internet, de atividades no computador, são pontos fortes para a não implementação da IE (...). Neste momento, o apoio específico do professor do laboratório se torna indispensável para que se atinja as sonhadas mudanças da IE".

A despeito de tal importância, muitas vezes o docente do LIE não está presente nas escolas. Odorico \cite{art:REF_ART_ODORICO}, ao investigar sobre os laboratórios de duas escolas da rede estadual de ensino de Minas Gerais, constatou que "um fator que pesa muito na utilização dos laboratórios é a presença na escola de um profissional da área de informática, já que muitos professores sentem-se inseguros e apresentam dificuldades para utilizar o computador. Entretanto nenhuma das instituições analisadas possui tal profissional para auxiliar os professores nas matérias específicas".

\subsection{O papel dos gestores}\label{sec:LABEL_CHP_REF_TEO_SEC_RES_GEST}

É notório que os gestores das escolas possuem muitos encargos, que oneram seu trabalho e o tornam desgastante e burocrático. Sobre isso, afirma Groll \cite{art:REF_TCC_GROLL} que "a gestão escolar se confunde, frequentemente, com administração, e não raro com Administração Científica, onde a burocracia impera. Mas o gestor escolar tem, além das tarefas administrativas e burocráticas, seu papel de professor".

Por conta dos encargos das funções de gestão, por vezes os gestores não se mostram interessados em trabalhar melhor o LIE no contexto pedagógico, pois os diversos problemas estruturais se tornarão novos encargos nas suas funções. Entretanto, no contexto do seu papel de professor, "é necessário que os dirigentes escolares atentem para o significado do trabalho de informatização como meio para a realização dos objetivos educacionais de natureza pedagógica, razão última da existência da escola" \cite{art:REF_TCC_GROLL}.

O papel do gestor no processo de implementação da IE na escola é explicitado por Groll \cite{art:REF_TCC_GROLL}: "os gestores, enquanto líderes, devem ser os encaminhadores rumo à tomada de consciência de toda comunidade escolar sobre a nova realidade que vivenciamos e que, portanto, a escola deve ser um ambiente aberto, democrático, flexível e naturalmente adequado para a promoção da inclusão digital e favorecer a criatividade e a inovação mediante o diálogo entre os vários componentes da comunidade".

% Chagas relata a respeito de cursos de formação para docentes do LIE na página 4 de seu artigo: "Quanto a sua Formação: Cursos de Especialização em Informática Educativa, era um investimento da PMF/Proinfo, para estes professores. Mas, a própria Prefeitura exigia o cumprimento destes profissionais nas atividades na escola".
% Isso pode ou não ser incluído no texto, neste capítulo ou em outro.