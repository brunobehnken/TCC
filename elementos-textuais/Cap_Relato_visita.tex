\chapter{Relato da Visita e da Tentativa de Aula}\label{chp:LABEL_CHP_REL}

No dia 31 de outubro de 2018, Bruno esteve em reunião com a professora Cassandra, na qual foi realizada a entrevista que foi relatada anteriormente. Nesta reunião, foi acordado que no dia 14 de novembro o laboratório seria utilizado pela Cassandra em parceria com Bruno e Filipe para ministrar uma aula no software \textit{JFractionLab}. Ficou acordado que Bruno e Filipe chegariam no CAp às 13h para realizar a instalação do software nas máquinas do laboratório, para que às 13h50min (início do segundo tempo de aula) a Cassandra pudesse levar os alunos ao laboratório e ministrar a aula.

No dia 14 de novembro, quarta-feira, às 13 horas, Bruno e Filipe estavam no CAp para realizar a instalação do software nas máquinas, entretanto, a bolsista que estava responsável pelo laboratório naquele horário, que deveria chegar também às 13h, não estava presente. Bruno e Filipe foram à DALPE e lá foram informados de que a única saída seria aguardar, pois não era possível entregar a chave do laboratório a uma pessoa que não possuísse vínculo com o CAp. Bruno e Filipe, então, aguardaram que a bolsista chegasse, o que só se concretizou às 13h45min, cinco minutos antes do segundo tempo, horário combinado com a Cassandra para trazer os alunos ao laboratório. Ao longo do tempo de espera, Bruno e Filipe estiveram sentados em frente à porta do laboratório, e puderam constatar que diversos alunos procuraram o laboratório para utilizá-lo, mas o encontraram vazio, com a porta trancada.

Na ocasião da chegada da bolsista, o laboratório foi aberto por ela e constatou-se que estava trancada a porta interna do laboratório, que dá acesso à área onde fica o computador utilizado pelos bolsistas e outros materiais relevantes. A bolsista responsável naquele momento disse que apenas um outro bolsista tinha a chave, mas que ele já havia ido embora. Relatou que a DALPE não possuía cópia da chave, e que seria necessário chamar um funcionário da manutenção.

Em seguida, a bolsista foi até o aparelho de ar condicionado e Bruno e Filipe perceberam que, logo abaixo do aparelho, havia a tomada onde este se encontrava ligado, e, apoiado na tomada, um copo de Guaravita aberto, completamente cheio de água. O ar condicionado, que havia sido deixado ligado enquanto o laboratório permaneceu trancado, estava com um vazamento de água, e o copo fora posicionado de modo a coletar a água que pingava; entretanto, por conta do tempo que o laboratório permaneceu trancado com o aparelho ligado sem ninguém para supervisioná-lo, o copo havia transbordado. A água que pingava e transbordava, então, havia molhado o caixonete da tomada na qual o aparelho de ar condicionado encontrava-se ligado naquele momento, bem como havia molhado o gabinete de um computador e uma das mesas dos computadores. Escorrendo pela parede para o chão, a água havia formado uma grande poça abaixo da mesa que fora molhada. A bolsista se mostrou chateada com a situação, reclamando, sozinha, quem alguém havia deixado o ar condicionado ligado ao sair do laboratório.

Também foi observado que o laboratório possuía algumas cadeiras avariadas, com encosto e/ou assento rasgados, bem como uma cadeira cujo encosto havia sido removido, deixando os ferros que davam suporte ao encosto expostos, aumentando assim o risco de acidentes, principalmente com as crianças que frequentam o laboratório. Foi observado ainda que havia grande número de caixas de papelão, de conteúdo desconhecido, empilhadas atrás da porta do laboratório; e que a impressora do laboratório estava quebrada, com um papel anexado a ela, escrito “Em manutenção. Favor não ligar”. Outra observação relevante é que o local onde o laboratório se encontra possui uma pilastra no meio da sala, o que dificulta o ministério de aulas.

Neste momento, a professora Cassandra chegou com os alunos do 5º (quinto) ano para a aula. Ao acomodar os alunos no laboratório, foi necessário pedir para que as crianças tomassem cuidado com a área que estava molhada, e instruí-las no sentido de que não seria possível utilizar os computadores daquela área. Após acomodar todos os alunos no laboratório, ela pediu para que Bruno e Filipe falassem com eles. Bruno então informou-os de que não seria possível utilizar o laboratório naquele momento pois ainda estava faltando instalar o programa que seria utilizado por eles, o que gerou grande frustração nos alunos, que apresentavam visível empolgação com a perspectiva de assistir uma aula no laboratório. Ao recolher os alunos para levá-los de volta à sala de aula, Cassandra informou que poderia voltar ao laboratório no 4º (quarto) tempo de aula, às 16h30min, caso o problema da instalação do software fosse solucionado em tempo hábil, ao que Bruno e Filipe se comprometeram em envidar esforços neste sentido.

Enquanto os alunos estavam no laboratório, o funcionário da manutenção que fora convocado para sanar o problema da porta interna trancada chegou, munido de uma escada de madeira e um cabo de vassoura. Devido à presença das crianças no laboratório, ele aguardou até que pudesse realizar seu trabalho. Após a retirada delas, posicionou a escada contra a porta e, subindo nela, comentou com a bolsista: “de novo essa porta, é?”. Questionada por Bruno e Filipe, a bolsista informou que não era a primeira vez que este problema acontecia; ao contrário, era um problema recorrente, e questionada acerca do porquê da DALPE não possuir uma cópia da chave, ela não soube explicar; questionado, o funcionário informou que o setor de manutenção não possui cópia daquela chave, mas também não deu explicações do porquê. A bolsista também informou que os seguranças do CAp, que guardam a chave do laboratório quando este não está em uso, também não possuem cópia da chave da porta interna. Uma vez no topo da escada, o funcionário pegou o cabo de vassoura e passou-o por uma abertura acima da porta, de modo que este alcançasse a maçaneta pelo lado de dentro. Após alguns minutos de esforço, o funcionário conseguiu abrir a porta por dentro e deixou o recinto levando a escada e o cabo de vassoura, retornando logo em seguida com um pano de chão e um balde, para secar a poça d’água formada pelo ar condicionado que estava pingando. Questionado se o problema era recorrente, ele disse que era bastante frequente que os bolsistas esquecessem de esvaziar o copo que coletava as gotas que pingavam e este transbordasse, formando assim a poça no chão e molhando os objetos ao redor do aparelho. A bolsista presente aproveitou a ocasião para esvaziar o copo. 

Logo em seguida, Bruno e Filipe questionaram a bolsista a respeito da instalação do \textit{JFractionLab}, ao que foram informados de que não seria possível realizá-la pois ela não possuía a senha de administrador das máquinas. A bolsista orientou-os no sentido de que a instalação de novos programas deveria ser realizada por meio de solicitação à DALPE, e que, caso a solicitação fosse aprovada, os bolsistas se encarregariam de realizar a instalação dos programas solicitados e de informar quando o laboratório estaria disponível para utilização com estes programas. Interpelada a respeito de quem seria responsável por realizar a avaliação técnica da solicitação de instalação de software, uma vez que a DALPE não possui funcionário com tal competência, a bolsista informou que outro bolsista, o que possui a senha de administrador, faria a avaliação e aprovação ou não da solicitação, bem como o procedimento de instalação no caso de aprovação.

Uma vez que ficou claro que a decisão estava na mão de um bolsista, Filipe, que possuía o telefone deste bolsista, perguntou se seria possível realizar a instalação do programa obtendo uma autorização por telefone, ao que a bolsista que estava no laboratório aquiesceu, com a restrição de que a instalação fosse realizada por Bruno e Filipe, alunos de computação, pois ela era aluna de um curso não relacionado à informática e não julgava possuir as competências técnicas necessárias.

Efetuada a ligação para o bolsista responsável pelas instalações de software e explicada a situação, a autorização foi concedida e a senha de administrador foi passada para a bolsista do horário, que a colocou em uma das máquinas para que a instalação fosse realizada. Bruno e Filipe realizaram este procedimento, entretanto, a instalação não foi concluída com êxito, pois o \textit{JFractionLab} precisa de um software adicional para funcionar: o \textit{Java}. Ao ser informada da situação, a bolsista disse que o \textit{Java} poderia ser instalado somente mediante autorização do outro bolsista. Em nova ligação telefônica, explicada a situação, o bolsista recusou o pedido de instalação, alegando que as máquinas não possuíam capacidade de hardware suficiente para rodar o \textit{Java}. Filipe questionou tal decisão, uma vez que Bruno já havia obtido um relatório do hardware da máquina à qual tinham acesso de administrador, e este relatório demonstrava a plena capacidade de hardware daquela máquina para atender os requisitos necessários à instalação do \textit{Java}; entretanto, o bolsista foi irredutível, solicitou que o telefone fosse passado para a bolsista do horário e a orientou no sentido de não permitir a instalação do \textit{Java}. Questionado acerca da presença na máquina de outros softwares instalados que possuíam requisitos superiores ao \textit{Java}, o bolsista não deu quaisquer explicações e se manteve irredutível.

Bruno e Filipe começaram então a avaliar alternativas que pudessem executar o \textit{JFractionLab} sem a presença do \textit{Java}. Inicialmente, pensaram em procurar o software em algum site de jogos, de modo que ele rodasse em um navegador, e, se não encontrassem, tentar hospedá-lo em algum servidor que possuísse o \textit{Java}; entretanto, esta alternativa rapidamente se revelou inviável, pois os softwares em \textit{Java} que rodam em navegadores dependem de um \textit{plug-in}, que por sua vez depende da instalação do \textit{Java}. Foi avaliada então a alternativa de realizar uma portabilidade automática de código fonte para uma outra linguagem, uma vez que o \textit{JFractionLab} é software livre e possui código aberto. Uma vez que o software estivesse em outra linguagem, seria possível executá-lo sem a presença do \textit{Java}. Entretanto, os serviços de portabilidade automática encontrados se limitavam a programas pequenos, a menos que fosse realizada a compra de pacotes \textit{premium}, e o \textit{JFractionLab} possui código fonte muito extenso, inviável de ser portado desta forma. Além disso, uma pesquisa revelou que os programas portados utilizando esses serviços em geral resultavam em programas com muitos erros, que precisavam ser consertados manualmente, o que levaria muito tempo e não daria qualquer garantia de que o software portado funcionaria da mesma forma que o original, tornando, então, esta alternativa inviável.

Diante da falta de alternativas, Bruno e Filipe informaram à professora Cassandra que não seria possível utilizar o laboratório naquele dia. Realizaram, então, a entrevista com a bolsista presente no laboratório e retiraram-se da escola mais ou menos às 17 horas.
