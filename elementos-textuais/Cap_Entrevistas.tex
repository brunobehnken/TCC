\chapter{Compreendendo o Funcionamento do Laboratório}\label{chp:LABEL_CHP_ENT}

\section{A Opção por Entrevistar os Atores}\label{chp:LABEL_CHP_ENT_SEC_OPC}

Para melhor compreender a dinâmica de funcionamento interno do CAp, como o LIE se insere nesse contexto, como o LIE funciona em termos de organização atualmente, quais projetos são desenvolvidos, como as aulas são ministradas, entre outras informações relevantes; optamos por realizar entrevistas com alguns atores que consideramos como fundamentais para o funcionamento efetivo de um LIE.

No caso do CAp, esses atores são a Direção Geral (DG), como gestora de recursos financeiros e da escola como um todo; a Direção Adjunta de Licenciatura, Pesquisa e Extensão (DALPE), como setor responsável pelos recursos pedagógicos da escola, inclusive pelo laboratório; a Profª. Cassandra, que, ao conhecer nosso trabalho, se interessou pela proposta e aceitou utilizar o laboratório com o nosso suporte; e dois bolsistas, cuja função é permanecer no laboratório e auxiliar as atividades desenvolvidas nele.

A partir das entrevistas foi possível observar a situação atual do CAp, entender a dinâmica de funcionamento do laboratório e tirar algumas conclusões sobre os problemas existentes atualmente. As entrevistas também subsidiaram o trabalho que viria a seguir, de tentativa de utilização pedagógica do LIE.

As seções seguintes trazem as perguntas que foram realizadas a cada ator e as conclusões que foram tiradas a partir das respostas que obtivemos. As transcrições completas de todas as entrevistas encontram-se em anexos.

\section{Entrevista com a Direção Geral}\label{chp:LABEL_CHP_ENT_SEC_DG}

Para a diretora geral do CAp-UFRJ, foram realizadas as seguintes perguntas:
\begin{itemize}
\item Você acha importante, ou ao menos pertinente, o uso do laboratório?
\item O que compete ao gestor no uso do laboratório de informática?
\item Como você, como gestora do CAp, considerando as limitações às quais está sujeita, poderia promover o uso do laboratório?
\item Como um gestor hipotético que disponha de todos os recursos que ele quiser poderia promover o uso do laboratório?
\end{itemize}

A partir da entrevista com a diretora geral foi possível perceber que o CAp é aberto à Informática Educativa e à promoção do laboratório como recurso didático do colégio, e que não oferece resistência à esta promoção; bem como foi possível perceber que as principais limitações são relacionadas ao espaço físico e à ausência do docente do LIE, e que essas limitações são conhecidas pela Direção da escola, que demonstrou inclinação a saná-las.

Foi ainda possível compreender que a gestão do laboratório é dividida entre a DG, responsável pela infra-estrutura, e a DALPE, responsável pela administração e pela parte pedagógica do laboratório.

A transcrição completa desta entrevista encontra-se no anexo A.

\section{Entrevista com a Direção Adjunta de Licenciatura, Pesquisa e Extensão}\label{chp:LABEL_CHP_ENT_SEC_DALPE}

Para a diretora adjunta de licenciatura, pesquisa e extensão do CAp-UFRJ foram realizadas as seguintes perguntas:

\begin{itemize}
\item{Quais são as propostas pedagógicas atuais para o uso do laboratório?}
\item{Você acha que uma proposta pedagógica unificada para todos, como sugestão, seria melhor do que propostas individuais?}
\item{Como é utilizado o laboratório?}
\item{Há quantos bolsistas?}
\item{Qual a viabilidade da realização de um projeto de extensão, por parte dos alunos da UFRJ, no laboratório?}
\end{itemize}

A partir da entrevista com a diretora foi possível perceber que o laboratório é mais utilizado para acolher aulas de professores que levam lá as suas turmas, e que estes ficam responsáveis pela elaboração das propostas pedagógicas, o que implica que não necessariamente a proposta elaborada implementará a Informática Educativa. O laboratório não possui projeto pedagógico próprio, mas há, por parte da DALPE, o interesse de elaborar um.

Foi possível perceber que o LIE possui 5 (cinco) bolsistas associados a ele, que o ocupam de 8h às 17h, de segunda à sexta-feira. Os alunos a partir do 6º (sexto) ano fundamental possuem livre acesso ao laboratório, e não há restrições quanto ao uso dos computadores, exceto por um jogo, que teve seu acesso bloqueado por estar causando problemas à ordem. Os bolsistas monitoram as atividade dos alunos no laboratório e auxiliam os professores quando estes ministram suas aulas no LIE.

A diretora relatou que a sala é inadequada, pois é pequena e possui uma pilastra no meio, o que desencoraja o uso do laboratório pelos professores; frisou que não há funcionário responsável pela parte de tecnologia no CAp, e que não há docente no LIE; e mostrou-se aberta à realização de um projeto de extensão no laboratório.

A transcrição completa desta entrevista encontra-se no anexo B.

\section{Entrevista com a Professora Cassandra}\label{chp:LABEL_CHP_ENT_SEC_CASS}

A entrevista com a Professora Cassandra, que dá aulas de matemática para o Fundamental I, foi composta de duas etapas. Na primeira etapa foram feitas as perguntas discriminadas a seguir, e na segunda foi apresentado o software \textit{JFraction Lab} para que ela avaliasse a pertinência de um plano de aula utilizando-o. 

Na primeira etapa foram realizadas as seguintes perguntas:

\begin{itemize}
\item{Você utiliza, ou já utilizou, o laboratório de informática do CAp?}
\item{Por que você não utiliza o laboratório de informática? Gostaria de utilizá-lo? Quais motivações te levariam a sair da sala de aula e levar os alunos ao laboratório?}
\item{Você acha que o uso do laboratório pode ajudá-la a atingir os objetivos da sua disciplina?}
\item{Você vislumbra algum objetivo pedagógico além dos objetivos da sua disciplina para o qual o uso do laboratório possa contribuir?}
\item{Você estaria disposta a nos ajudar a construir uma proposta didática para o uso do laboratório na sua disciplina?}
\item{Se nós elaborarmos uma proposta pedagógica multidisciplinar envolvendo a sua disciplina, você aceitaria participar?}
\item{O que podemos elaborar como proposta pedagógica para o laboratório dentro da sua disciplina ainda para esse ano?}
\end{itemize}

A partir das respostas dadas a estas perguntas, foi possível concluir que a Profª. Cassandra nunca utilizou o laboratório porque nunca parou para planejar uma aula nele, mas que esta vontade está presente, pois ela se sente atraída pela parte gráfica, com representações geométricas e visualizações, utilizando, por exemplo, o \textit{Google Maps}, software que ela vislumbra sendo utilizado de modo multidisciplinar.

Na segunda parte da entrevista, a Profª. Cassandra relatou que estava, naquela época, trabalhando o conteúdo de frações com a sua turma de 5º (quinto) ano, e que seria oportuno utilizar um software que representasse as frações em gráficos como o de pizza, e permitisse que eles treinassem operações básicas, como simplificações. 

Foi apresentado então o software \textit{JFraction Lab}\footnote{\textit{Software} educacional desenvolvido com a tecnologia \textit{Java} que trabalha o conteúdo de frações utilizando recursos visuais, como gráficos, e possibilitando ao aluno praticar com exercícios de operações básicas, como soma e simplificação de frações.}, que atende aos requisitos colocados. A professora se interessou pelo software e se comprometeu a elaborar um plano de aula no laboratório envolvendo o uso deste, conosco acompanhando a aula e ajudando-a a ministrá-la.

Na ocasião da entrevista já deixamos marcado o dia e o horário em que a aula seria ministrada, e ficou acordado que chegaríamos cinquenta minutos antes do início da aula para instalar o software nas máquinas do laboratório.

A transcrição completa desta entrevista encontra-se no anexo C.

\section{Entrevista com a bolsista Ingrid}\label{chp:LABEL_CHP_ENT_SEC_ING}

Para a bolsista Ingrid, que trabalha no LIE, foram realizadas as seguintes perguntas:

\begin{itemize}
\item{Qual o seu curso na UFRJ? Há quanto tempo você está trabalhando como bolsista do laboratório? Quais são os seus horários aqui?}
\item{Você acha que muitas crianças vêm ao laboratório? Quantas, mais ou menos?}
\item{Em geral, o que elas fazem no computador?}
\item{Como funciona o processo de supervisão das crianças no laboratório?}
\item{Você ajuda os professores que trazem seus alunos para realizar tarefas no laboratório?}
\item{Quais as principais demandas que surgem no dia a dia?}
\end{itemize}

A partir da entrevista com a bolsista foi possível perceber que de 20 (vinte) a 25 (vinte e cinco) alunos visitam o LIE todas as tardes. As principais atividades deles no laboratório são fazer trabalhos e assistir vídeos no \textit{YouTube}, entretanto a maioria dos vídeos possui áudio, e o LIE possui apenas um fone de ouvido para disponibilizar aos alunos. Além disso, alguns alunos gostariam de acessar jogos \textit{on-line}, mas esse acesso foi bloqueado.

Por conta do uso indevido do laboratório, foram colados avisos de relacionamento com o espaço nas paredes. Foi relatado que esses avisos auxiliam na supervisão dos alunos no laboratório.

Foi possível saber também que os professores realizam o agendamento de suas aulas no laboratório por meio de uma planilha que fica disponível \textit{on-line} para todos os professores, e que as principais demandas que eles apresentam aos bolsistas são de ordem técnica -- os planos de aula, bem como a sua execução, são realizados sem a ajuda dos bolsistas.

A transcrição completa desta entrevista encontra-se no anexo D.

\section{Entrevista com o bolsista Pedro}\label{chp:LABEL_CHP_ENT_SEC_PED}

Para o bolsista Pedro, que trabalha no LIE, foram realizadas as seguintes perguntas:

\begin{itemize}
\item{Qual o seu curso na UFRJ? Há quanto tempo você está trabalhando como bolsista do laboratório?}
\item{Você acha que muitas crianças vêm ao laboratório? Quantas, mais ou menos?}
\item{Em geral, o que os alunos fazem no computador?}
\item{Como funciona o processo de supervisão das crianças no laboratório?}
\item{Você ajuda os professores que trazem seus alunos para realizar tarefas no laboratório?}
\item{Quais as principais demandas que surgem no dia a dia?}
\item{Você acha que o laboratório atrairia mais os alunos se oferecesse mais coisas a eles?}
\item{Quais são os principais problemas que você identifica no laboratório?}
\item{Você está se formando em Educação Física. Como docente deste componente curricular, você usaria o laboratório? Como?}
\item{Ouvimos da bolsista Ingrid que certa vez tentaram instalar o \textit{LoL}\footnote{\textit{League of Legends}. Jogo eletrônico da categoria MOBA (\textit{Multiplayer Online Battle Arena}) desenvolvido pela empresa \textit{Riot Games}. É muito popular entre os jovens brasileiros.} em um dos computadores. Esse é um jogo muito famoso, o maior expoente do \textit{e-sports}, que é uma área que se integra com a Educação Física, sua área de formação. Considerando que a procura por jogos aqui no laboratório é grande, você o utilizaria para fazer algo na área de \textit{e-sports} com os alunos do CAp?}
\end{itemize}

A partir da entrevista com o bolsista foi possível perceber que o número de alunos que frequenta o laboratório varia durante o ano, e é mais expressivo no início dos períodos letivos, e menos expressivo no final. Como a entrevista foi realizada em novembro, naquela época a frequência média de alunos era de 10 (dez) por dia.

Foi possível perceber também que a maior demanda dos alunos é por vídeos no \textit{YouTube} e jogos \textit{on-line}, o que condiz com o que foi relatado pela bolsista Ingrid. Há ainda uma grande demanda por impressões de trabalhos, da parte dos alunos, e por impressões de provas, da parte dos professores. Os professores que dão aula no laboratório por vezes apresentam também a demanda de instalação de software e de testes de vídeos a serem exibidos, a fim de se assegurarem de que tudo funcionará corretamente no momento da aula.

Foi relatado ainda que a função do bolsista é de zelar pelo uso do laboratório e que a supervisão dos alunos consiste em observá-los e intervir caso estejam fazendo uso inadequado do espaço.

Assim como relatado pela bolsista Ingrid, o bolsista Pedro também informou que os professores realizam o agendamento da aula utilizando uma planilha \textit{on-line}, entretanto, foi dito que poucos professores utilizam o laboratório, uma estimativa de quatro ou cinco, e que são sempre os mesmos. Estes muitas vezes não usam os recursos computacionais do laboratório, apenas a televisão como recurso multimídia. Neste contexto, o bolsista informou que sente falta de um \textit{workshop} voltado para o uso dos recursos computacionais do laboratório, pois a televisão é o recurso mais utilizado pelos professores.

Quanto aos problemas, conforme relatado por outros atores, o bolsista informou que a pilastra no meio da sala acaba sendo problemática, pois o professor não possui uma visão completa da sala. O laboratório é pequeno, e com turmas grandes se torna necessário que os alunos dividam os computadores, e isso, por vezes, gera alguns atritos entre eles. O ar-condicionado que pinga dentro do laboratório é o principal problema atualmente.

Além disso, foi colocado como um problema a manutenção do laboratório ficar a cargo dos bolsistas, principalmente porque há equipamentos que eles não sabem consertar, como a impressora, que havia quebrado há algumas semanas e permanecia assim, aguardando manutenção especializada -- enquanto isso, as demandas por impressão, recorrentes no laboratório, não estavam sendo atendidas.

O bolsista relatou ainda o desejo de utilizar um \textit{kinect} no laboratório, junto com um projetor e um \textit{X-Box}, pois estes equipamentos seriam relevantes para o ensino da Educação Física. Mostrou ainda a vontade de desenvolver uma proposta pedagógica no LIE que utilizasse os \textit{e-sports}, pois os alunos demonstram uma forte aderência a eles.

A transcrição completa desta entrevista encontra-se no anexo E.

\section{O que Percebemos e Concluímos}\label{chp:LABEL_CHP_ENT_OBS_CONC}

As entrevistas que realizamos com os diversos atores ligados ao LIE foram fundamentais para termos uma melhor percepção do \textit{modus operandi} do CAp e do laboratório em si. A partir delas foi possível observar e concluir diversas coisas.

O LIE do CAp não possui o seu próprio docente, ator classificado como indispensável por Chagas \cite{art:REF_ART_CHAGAS}; entretanto, esta função é, até certo ponto, desempenhada por bolsistas da própria UFRJ. Portanto, conclui-se que, apesar da ausência do docente do LIE no laboratório do CAp, a situação não é tão precária quanto a que Odorico \cite{art:REF_ART_ODORICO} encontrou nas escolas que visitou na rede estadual de Minas Gerais, pois estas não possuíam nem mesmo bolsistas ou estagiários.

Entretanto, pudemos perceber que a presença dos bolsistas não é suficiente para fomentar a implementação da Informática Educativa, fato evidenciado nas duas entrevistas realizadas com os gestores (DG e DALPE), que demonstraram desejo de que o laboratório fosse mais utilizado para fins pedagógicos, tanto no contexto de cada disciplina isolada quanto no contexto do colégio como um todo.

Fica evidente que os bolsistas não têm possibilidade de desenvolver projetos na área de IE, ainda que estes queiram fazê-lo, e esta vontade se mostra presente, conforme evidenciado pelo bolsista Pedro no final de sua entrevista. Isto se dá pois os bolsistas não possuem vínculo empregatício com o CAp, e não podem ser responsabilizados por projetos ou por eventuais problemas que venham a acontecer no laboratório.

Por conta da ausência do docente do LIE, os bolsistas acabam também assumindo encargos que não deveriam (e talvez nem poderiam) ser seus, tais como a manutenção de hardware e software das máquinas no laboratório, bem como a responsabilidade de decidir se um software deve ou não ser instalado no computador (encargo evidente no relato da visita que realizamos) ou se um site ou software deve ser acessado ou não pelos alunos.

Neste contexto, identificamos uma contradição nas entrevistas. A Profª. Isabel, gestora da DALPE, relatou que o uso de jogos era liberado no laboratório, com exceção de um jogo específico, entretanto, a bolsista Ingrid relatou que os jogos \textit{on-line} estavam com acesso bloqueado, fato que pôde ser comprovado por nós naquela data. Surgiu então uma dúvida a respeito da sinergia entre os bolsistas e a DALPE.

Apresentamos à Profª. Isabel a informação de que os jogos \textit{on-line} estavam bloqueados e perguntamos se o bloqueio que fora realizado havia sido autorizado, ao que ela redarguiu que apenas um jogo deveria ser bloqueado pelos bolsistas, mas que ela não sabe como funciona o processo de bloqueio a esse jogo, e que há a possibilidade de que a solução técnica encontrada pelos bolsistas para este problema acabe por bloquear também o acesso a outros jogos.

Identificamos, assim, que a falta de preparo técnico dos bolsistas compromete os objetivos estabelecidos pela DALPE, uma vez que há melhores soluções do ponto de vista técnico para o bloqueio do acesso a um jogo; e que a sinergia entre estes dois atores fundamentais não é tão grande quanto desejável, uma vez que não foi informado à DALPE que o procedimento de bloquear todos os jogos \textit{on-line} seria tomado para efetivar o bloqueio a um único jogo.

Outra observação importante é a respeito da nomenclatura do laboratório. A Profª. Isabel se referiu a ele como "LIG", que significa Laboratório de Informática de Graduação, nome que não reflete com exatidão os objetivos pedagógicos do ensino básico. Não temos por objetivo discutir assuntos pouco relevantes, como a nomenclatura adotada; entretanto, este é um indício de que talvez a Informática Educativa não esteja muito bem definida para a gestão do CAp.

No mais, as entrevistas permitiram o conhecimento da dinâmica de funcionamento do laboratório: ele fica aberto aos alunos, que podem entrar e sair livremente se forem do Fundamental II ou do Ensino Médio, e podem usar o computador da maneira que quiserem, respeitadas algumas restrições. Se um professor quiser ministrar uma aula no laboratório, é necessário reservá-lo, e esta reserva é realizada através de uma planilha \textit{on-line} que fica disponível para os professores. O suporte técnico para a aula pode ser solicitado pelo professor e é oferecido pelos bolsistas. Além dos computadores, o laboratório possui um aparelho de televisão.
