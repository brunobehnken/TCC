\chapter{Entrevistas}\label{chp:LABEL_CHP_ENT}

Neste capítulo são apresentadas as transcrições das entrevistas que foram realizadas com os diversos atores relacionados ao LIE no CAp-UFRJ. A partir delas, foi possível observar a situação atual do CAp e tirar algumas conclusões sobre os problemas existentes atualmente. As perguntas realizadas estão grafadas em itálico.

\section{Entrevista com a Direção Geral}\label{chp:LABEL_CHP_ENT_SEC_DG}

A presente entrevista foi realizada com a Professora Maria Cristina Miranda da Silva, Diretora Geral do CAp-UFRJ, no dia 18 de outubro de 2018.

\textit{Você acha importante, ou ao menos pertinente, o uso do laboratório?}

O uso do laboratório é importante no contexto escolar e pedagógico para que as disciplinas tenham apoio nos seus conteúdos. Cito meu próprio caso como exemplo, por ser professora de artes visuais: há várias questões em que pode se utilizar os computadores, usando softwares para pesquisas sobre arte ou para a própria “atividade fim”, a criação da arte visual.

O laboratório poderia ser um espaço em que o estudante ficasse no contraturno, para fazer trabalho ou realizar pesquisa, utilizando-o como apoio acadêmico. Entretanto, o laboratório não funciona como seria desejável. Dentre os diversos motivos, identifico dois centrais:

Não há funcionário específico para o laboratório. Trabalhamos com bolsistas, e esse trabalho é difícil, pois, por serem estudantes, os bolsistas não têm autoridade de funcionário para administrar esse laboratório.

Não existe uma cultura entre o corpo docente para utilizar o laboratório, pois o professor é co-responsável pela forma de trabalho. Ele não pode, por exemplo, ficar só com um aluno e deixar os outros sem supervisão. Uma cultura de regras e formas de utilização precisa ser pensada, e o projeto de vocês pode ajudar a construir essa cultura. Observamos problemas relacionados ao uso do laboratório, como alunos que escrevem que são fascistas no plano de fundo do computador, dentre outros. Temos 720 alunos no CAp, em média 30 por turma, não sei exatamente a capacidade do laboratório, mas ele não comporta todos os alunos de uma turma, e precisa ser trabalhado com frações dela; isso também é um problema na criação dessa cultura de utilização do laboratório.

\textit{O que compete ao gestor no uso do laboratório de informática?}

Compete ao gestor zelar pelo equipamento que está ali, pois ele é caro. Primeiro a compra, depois o zelo, e depois a manutenção. Compete ao gestor o acompanhamento do uso do equipamento, e também compete a ele solicitar os funcionários necessários. Considerando o “gestor” como “o grupo coletivo de trabalhadores com a responsabilidade da tarefa de gerir o laboratório”, também compete fazer a escolha dos programas que serão disponibilizados. No caso do CAp, compete à Direção Geral fazer a compra dos equipamentos e os pedidos de funcionários, bem como pedidos de infra-estrutura, como ar condicionado, mesas, etc. A DALPE é a que fica responsável pela “administração” do espaço. A mediação entre a administração e os professores e estudantes é feita pela DALPE e DAE. Não quer dizer que é assim sempre, em todos os lugares, nem que essa é a melhor forma de gestão, mas foi assim que o CAp se estruturou. 

\textit{Como você, como gestora do CAp, considerando as limitações às quais está sujeita, poderia promover o uso do laboratório?}

A questão de não haver um funcionário específico para o laboratório é fundamental, e compete à Direção Geral. O trabalho com os bolsistas é muito complicado, pois não há autoridade para o bolsista.

\textit{Autoridade para o bolsista em que sentido?}

Os bolsistas são jovens, estudantes, que não podem ser responsabilizados por coisas graves. O bolsista não é um funcionário, mas sim um estudante. Por exemplo, o CAp é um campo de estágio que recebe licenciandos, mas esses licenciandos não podem ficar sozinhos com os alunos. É necessário ter um funcionário da escola que se responsabilize por esse contato. O bolsista \textbf{auxilia} o processo, mas não pode ser responsável por ele. Isso é um problema na liberação do laboratório em tempo vago, por exemplo. No caso em que um professor precise faltar e não dê sua aula, as crianças não podem ir para o laboratório porque o bolsista não pode se responsabilizar pelo uso dele. É necessário ter um professor ou funcionário presente. Temos bolsistas cuidando do laboratório enquanto o funcionário cuida da turma, essa é a separação.

\textit{Como um gestor hipotético que disponha de todos os recursos que ele quiser poderia promover o uso do laboratório?}

Num cenário ideal a gente teria um laboratório maior, que comportasse a turma toda, com mais máquinas, uma por aluno, e com pelo menos dois funcionários, que pudessem atuar um na orientação do bolsista e outro na recepção do estudante, que fossem especialistas no assunto, formados em educação ou informática no ensino. Esses funcionários poderiam pensar o laboratório, fazer dele um projeto grande na escola, um projeto da escola como um todo, independente, montado pelos funcionários, com conhecimento pedagógico e de computação, com horário separado na grade. Alguém que use o laboratório não deve estar preocupado com computador que não liga, com puxar fio de tomada, essa pessoa precisa só utilizar o equipamento, sem se preocupar com a manutenção. Precisa ter um educador com essa perspectiva. O professor só vai no laboratório com os alunos para fazer uma pesquisa ou digitar algo. É preciso criar essa cultura de uso do laboratório, é preciso uma equipe que pense as finalidades, que divida a responsabilidade do laboratório nessa perspectiva técnica e didática.

\textit{Na sua graduação como professora, você teve contato com a informática? E com a informática voltada para a educação?}

Não tive disciplina de uso do computador na graduação porque fiz graduação há muito tempo, não existia computador pessoal naquela época (\textit{risos}); mas tive contato com o computador por fora. Fiz, inclusive, um curso no qual aprendi a programar em \textit{Basic}, mas cartão perfurado não era a minha praia. Gosto de artes visuais.
Hoje eu enxergo o computador como uma extensão do corpo, sabe? Eu fico pensando, se eu for presa, o que vou fazer na cadeia sem celular e computador? Se eu precisasse sobreviver, não me preocuparia com a comida, mas com a tecnologia (\textit{risos}).

\textit{E hoje, você sabe como é esse contato com a informática voltada para a educação nas graduações?}

Não sei, e fico curiosa para saber. Saber como a informática é tratada em cursos de pedagogia e licenciatura, se há disciplina de informática voltada para a educação, se há interesse em pensar nas mídias e no ensino. São duas coisas, uma é você aprender a utilizar a tecnologia como mediador, e outra é trabalhar a atividade fim, a educação. Para encerrar, deixo vocês com uma pergunta a ser pensada: atualmente as crianças “já nascem” sabendo usar a tecnologia. Hoje ainda cabe ensinar uma criança a mexer no computador?

\section{Entrevista com a Direção Adjunta de Licenciatura, Pesquisa e Extensão}\label{chp:LABEL_CHP_ENT_SEC_DALPE}

A presente entrevista foi realizada com a Professora Isabel Van Der Ley Lima, Diretora Adjunta de Licenciatura, Pesquisa e Extensão do CAp-UFRJ, no dia 18 de outubro de 2018.

\textit{Quais são as propostas pedagógicas atuais para o uso do laboratório?}

Primeiro vou contar uma história: o Laboratório de Informática (LIG) fica ligado à DALPE, pelo regimento do CAp. Até o ano passado a Profª. Izabel estava aqui, e tinha conhecimentos tanto técnicos quanto pedagógicos. Eu e a Profª. Marilane (a outra diretora da DALPE) não temos essa familiaridade tão grande.

Hoje o uso do laboratório é basicamente para acolher as aulas. As propostas pedagógicas são desenvolvidas pelos professores dos setores, e às vezes elas não têm muito a ver com o uso da informática. Por exemplo, hoje um professor de línguas usou o laboratório para passar um filme, e não usou os computadores; então o uso é mais do espaço do que do laboratório, o que gera uma certa insatisfação da parte dos bolsistas.

A coordenação do LIG não tem proposta pedagógica específica, porém temos a proposta de que os bolsistas familiarizem os alunos no sentido de compreender o laboratório como espaço acadêmico público. É muito frequente que alunos, principalmente do Fundamental II, entrem no LIG e causem danos aos materiais, tanto físicos, como tirar teclas do teclado e brincar de ficar empurrando as cadeiras, quanto danos digitais, pois às vezes eles alteram a configuração do computador, e essa alteração impede o uso de outros colegas, aí os bolsistas precisam descobrir o que foi modificado para reverter; então tem sido trabalhado que os bolsistas conversem com os alunos no sentido de conscientizar sobre o uso do laboratório. Sugiro que vocês entrevistem o bolsista Marcos, o horário dele é segunda e quarta de manhã. Eu brinco dizendo que ele é o “coordenador dos bolsistas”, pois é o que tem mais acúmulo de experiências.

Quanto ao uso dos jogos, ele é livre, mas há um jogo específico que estava gerando muita discussão. Não sei qual jogo é, mas tem conteúdo violento, e os alunos ficavam muito exaltados. Isso interferia no uso físico do laboratório, então, depois de muita discussão, acabamos decidindo por bloquear esse jogo específico, mas o uso dos jogos em geral é livre. O que a gente tem trabalhado nesse sentido é colocar alguns avisos de relação com o laboratório como espaço acadêmico.

As propostas pedagógicas além disso ficam a cargo dos setores pedagógicos que utilizam o laboratório. Na minha opinião, a matemática é o setor que mais utiliza o laboratório com o intuito de pensar propostas pedagógicas para o ensino associadas à questão da informática. O professor Fernando Velar tem um projeto de pesquisa e extensão no laboratório com os alunos da educação básica, acho que o professor Cléber utilizou o laboratório para o mestrado também.

\textit{A matemática tem uma visão mais pedagógica, mas como as outras áreas utilizam o laboratório?}

Os bolsistas podem informar melhor quais áreas usam e com quais finalidades. Aqui na DALPE a gente não pede para o professor registrar o objetivo da proposta, pedimos apenas para ele marcar a aula, então quem percebe melhor essa questão são os bolsistas, que estão ali no dia a dia. Pelo que eles me dizem, a área que mais utiliza o laboratório com objetivo pedagógico ligado à tecnologia é a matemática. As outras áreas usam mais o espaço do laboratório do que o laboratório em si, como foi o caso do professor que usou o laboratório hoje para passar filme.

\textit{Você acha que uma proposta pedagógica unificada para todos, como sugestão, seria melhor do que propostas individuais?}

O LIG poderia propor usos pedagógicos mais gerais. Uma questão forte agora é a inclusão. Temos um aluno que é autista, e que faz muito uso do LIG com a professora que o acompanha, de educação especial. Seria legal pensar em como essa professora trabalha, para poder expandir esse trabalho. Estamos montando uma sala de recursos, que ficará ao lado do LIG, e estamos pensando na questão do uso da informática como recurso para inclusão.

Seria bom termos mais propostas pedagógicas, pois muitos professores não fazem uso do laboratório pois não relacionam o uso da tecnologia às aulas. Aqui no CAp os setores pedagógicos têm muita autonomia na construção dos seus trabalhos pedagógicos, embora exista o projeto da escola em si. A gente poderia propor uma proposta pedagógica maior, mais geral, embora isso não seja feito; mas mesmo com essa proposta depende de cada setor comprar ou não a ideia. As propostas precisam ser mais uma sugestão, e não algo que comprometesse o trabalho.

Gostaria de abordar também uma questão estrutural: o laboratório é pequeno e tem uma coluna no meio do caminho, o que dificulta para dar aula; então alguns professores optam por não ir pela dificuldade do espaço. O laboratório até comporta todos os alunos, mas turmas menores, ou frações de uma turma, costumam funcionar melhor. Os professores de línguas usam mais o laboratório porque suas turmas são menores.

\textit{Como é utilizado o laboratório?}

Todos os horários têm um bolsista, e o trabalho dele é supervisionar o uso do laboratório pelos alunos quando não está tendo aula, e oferecer suporte ao professor quando há aula. Na nossa percepção, os alunos do Fundamental II usam muito para jogar, e os do Ensino Médio usam mais para fazer trabalhos. Houve uma discussão sobre uso do laboratório pelo Fundamental I, para que eles o utilizassem apenas sob supervisão de professores (sem uso livre), mas o Fundamental II e o Ensino Médio têm acesso tanto livre quanto com professor. Os horários do recreio são os que têm maior procura.

\textit{Há quantos bolsistas?}

Atualmente temos 5 (cinco) bolsistas. Eles trabalham 8 (oito) horas, divididas em dois turnos. Podem ser duas manhãs, duas tardes, ou uma manhã e uma tarde.

\textit{Qual a viabilidade da realização de um projeto de extensão, por parte dos alunos da UFRJ, no laboratório?}

Seria ótimo um projeto de extensão. A prioridade do uso é das aulas, depois do uso livre, então, acontecendo um projeto, seria necessário reservar o uso do laboratório, para não haver conflito com as aulas, pois o professor do CAp é apegado às suas aulas e é difícil mudar planejamentos, então não pode haver conflitos, mas esse é o único requisito. A questão é marcar com antecedência para garantir o uso do laboratório.

Se vocês conversarem com os bolsistas, podem identificar o que é mais necessário, quais as maiores demandas, porque a minha percepção é indireta, através de reuniões eventuais entre a DALPE e os bolsistas, para avaliar determinadas situações.

\textit{Podemos participar dessa reunião?}

Ainda não temos dia e horário para a reunião, mas a próxima deve ser na semana que vem. Vocês podem deixar seus contatos que eu aviso.

\textit{Na sua graduação como professora, você teve contato com a informática? E com a informática voltada para a educação?}

Olha, meu primeiro computador foi um 486, vocês não devem nem saber o que é isso! (\textit{risos}). Eu fiz cursos de DOS e de Windows 3.1, mas fora da graduação. A UFRJ tinha muito pouco dinheiro no final da década de 90, quando eu fiz minha graduação, e a parte de informática era muito precária, então não houve contato com a informática, embora algumas matérias demandassem o uso do computador, como análise de dados, estatística, etc. Hoje eu acho que o LIG é mais funcional na Biologia da UFRJ. Agora, disciplina de informática voltada para a educação, nenhuma. Não houve diálogo entre informática e educação, mas hoje há disciplinas nessa linha no NUTES, lá no CCS, são optativas. Eles são legais, abertos ao diálogo, e fazem trabalhos com escolas.

Dentro da UFRJ, já como docente, eu tive contato com a informática na educação por causa do Prof. Fernando. Fomos ao NCE para verificar possibilidades de relacionar educação com tecnologia, mas institucionalmente nunca houve incentivo. Hoje a formação ainda tem pouco diálogo com isso, são mais iniciativas pontuais dos professores. Os currículos que eu tenho visto não oferecem esse diálogo.

Estou trabalhando no complexo de formação de professores da UFRJ, que propõe uma reestruturação dessa formação, e mesmo assim não vejo muitas discussões sobre tecnologia. Acho que temos tantos problemas que são tão mais básicos, como conciliar o horário do aluno, por exemplo; há cursos de licenciatura que não dialogam com o local onde ele faz estágio. Isso é uma coisa simples de ser feita e nós temos problemas, então a tecnologia acaba ficando um pouco de lado.

Há também uma questão estrutural da universidade. Por exemplo, não existe rede de informática aqui na DALPE, tudo foi feito na base do “puxadinho”. Recorri à TIC/UFRJ, e eles se mostraram abertos a fazer um projeto de rede, mas é necessário ter a planta baixa do local. Fui no setor responsável por isso, mas há dificuldade para obter essa planta pois o prédio não é da UFRJ, é cedido pela prefeitura.

O CAp não tem funcionário responsável pela parte de tecnologia, então outra coisa que eu verifiquei com a TIC/UFRJ foi a possibilidade de ter, mas não há disponibilidade, pois a própria TIC/UFRJ tem um número pequeno de funcionários para atender à Universidade como um todo, e não há como deslocar um funcionário para uma unidade. É difícil ter acesso a coisas em geral. Aqui no CAp só temos 3 (três) salas de aula com \textit{Datashow}, que é um recurso simples, só projeta uma imagem. Só há um funcionário relacionado à parte de áudio-visual.

Faltam profissionais que tenham formação não apenas na área de informática, mas na área de educação também. Essa é uma questão visível na seleção dos bolsistas: há bolsistas de Ciência da Computação que são ótimos técnicos, mas têm dificuldade no diálogo educacional; e há bolsistas com formação pedagógica que não sabem usar o computador. Por conta disso, tentamos estabelecer um diálogo entre eles para tentar complementar a formação de ambos. Temos uma bolsista que faz Licenciatura em Letras e tem conhecimentos de informática, mas por fazer disso uma fonte de renda e não por ter formação acadêmica para tal.

Eu vejo que há pouca interface de diálogo entre a tecnologia e a educação, tanto na UFRJ quanto em outros lugares, como a PUC-RJ, onde eu fiz doutorado, e é um trabalho muito necessário.

\textit{A PUC-RJ possui um Colégio de Aplicação também, não é?}

Sim, mas o colégio não é da PUC-RJ, é um convênio. Inclusive, eu tenho uma amiga que trabalha lá e tem uma disciplina que trabalha com tecnologias.

Eu fiz doutorado em educação na PUC-RJ. Quando eu estava saindo de lá, eles estavam contratando um professor para dialogar com a parte de educação e tecnologia, mas não sei se foi contratado mesmo, porque realmente falta gente que faça esse diálogo.

% OBS: o fato do laboratório ser chamado de LIG e não de LIE é um indicativo de desconhecimento a respeito da informática educativa.

\section{Entrevista com a Professora Cassandra}\label{chp:LABEL_CHP_ENT_SEC_CASS}

A presente entrevista foi realizada com a Professora Cassandra, que dá aulas de matemática para o Fundamental I, no dia 31 de outubro de 2018.

\textit{Você utiliza, ou já utilizou, o laboratório de informática do CAp?}

Não.

\textit{Por que você não utiliza o laboratório de informática? Gostaria de utilizá-lo? Quais motivações te levariam a sair da sala de aula e levar os alunos ao laboratório?}

Nunca utilizei o laboratório como professora, e nunca parei para pensar no porquê. Tenho o desejo de utilizar, mas nunca parei para planejar de fato uma aula lá. Tenho esse desejo porque sei dos recursos disponíveis no laboratório, e sei que eles favorecem a aprendizagem e tornam mais lúdico o processo de ensino. Ano passado eu estava como professora do 1º (primeiro) ano, são crianças muito pequenas, e por isso é complicado de levá-las ao laboratório; mas esse ano, pela primeira vez, estou com o 5º (quinto) ano, então esse desejo surgiu. Ano que vem, se eu continuar com o 5º (quinto) ano, certamente vou procurar usar o laboratório.

\textit{Você acha que o uso do laboratório pode ajudá-la a atingir os objetivos da sua disciplina?}

Não parei para pensar sobre isso, mas sei que sim. De maneira mais geral: além da questão do lúdico, o que chama a minha atenção é a possibilidade da representação gráfica. Representações geométricas, coisas observáveis, para medição de tempo ou de quantidades. O \textit{Google Maps} pode ser usado para medir distâncias e para noções geométricas, por exemplo. O que me atrai mais é a parte de visualização.

Quanto ao uso do recurso tecnológico, cheguei a indicar para os meus alunos um aplicativo de celular que auxilia no aprendizado da tabuada, mas é apenas um joguinho. Todos eles tem celular, todos eles baixam joguinhos, né? Por que não fazer disso algo útil?

\textit{Você vislumbra algum objetivo pedagógico além dos objetivos da sua disciplina para o qual o uso do laboratório possa contribuir?}

Bom, o \textit{Google Maps} dá para usar com várias disciplinas, tanto na parte de localização quanto com noções geográficas, de organização do espaço; talvez ver o CAp pelo satélite, e localizar a quadra, por exemplo. Outra coisa bacana e multidisciplinar é produção de vídeo.

Entretanto, não dá para pensar no objetivo pedagógico no sentido de ocupar a aula para que eles aprendam a usar o recurso. Teria que ser algo mais prático, com foco nos objetivos da disciplina.

Dei aos meus alunos recentemente uma tarefa de casa: tirar ou encontrar fotos da cidade em tamanho 10 cm x 10 cm para serem impressas e comporem um portfólio. O laboratório poderia ser usado como espaço de pesquisa para isso, e isso poderia ser uma proposta mais geral, no sentido de orientar uma pesquisa multidisciplinar na internet. Se tivesse espaço nas aulas para orientar e conduzir esse tipo de pesquisa, ensinar aos alunos como pesquisar, conversar sobre quais sites são confiáveis para fazer uma pesquisa, quais informações têm \textit{copyright}, etc.

\textit{Você estaria disposta a nos ajudar a construir uma proposta didática para o uso do laboratório na sua disciplina?}

Claro!

\textit{Se nós elaborarmos uma proposta pedagógica multidisciplinar envolvendo a sua disciplina, você aceitaria participar?}

Sem dúvida!

\textit{O que podemos elaborar como proposta pedagógica para o laboratório dentro da sua disciplina ainda para esse ano?}

Meus alunos do 5º (quinto) ano estão vendo frações e números decimais agora. Seria legal um programa que eles pudessem observar as frações em gráficos de pizza e treinar operações como simplificação.

\textit{O} JFraction Lab\footnote{\textit{Software} educacional desenvolvido com a tecnologia \textit{Java} que trabalha o uso de frações.} \textit{parece ser a alternativa perfeita. O que acha dele? (Neste momento o} software \textit{foi apresentado à professora).}

É excelente! Adorei! Tem diversas possibilidades para abordar com os alunos. Vou baixar em casa, no meu computador, para explorar mais e elaborar uma aula com ele. Se eu tiver tempo, coloco o plano de aula no papel e envio para vocês. Na quarta feira eu tenho dois tempos de aula, de 13h às 14h40m. Podemos já deixar pré-marcada nossa aula em dia e horário provisório. Que tal 14 de novembro, das 13h50m as 14h40m? Assim, no primeiro tempo, de 13h às 13h50m, eu estarei com a turma em sala, e no segundo os levarei ao laboratório.

\textit{Para nós está excelente! Podemos chegar às 13h para instalar o programa em todas as máquinas e preparar o laboratório para a aula.}

Perfeito então.

\section{Entrevista com a bolsista Ingrid}\label{chp:LABEL_CHP_ENT_SEC_ING}

A presente entrevista foi realizada com a bolsista Ingrid, que trabalha no LIE, no dia 14 de novembro de 2018, dentro do próprio laboratório.

\textit{Qual o seu curso na UFRJ? Há quanto tempo você está trabalhando como bolsista do laboratório? Quais são os seus horários aqui?}

Eu faço Letras em Latim. Estou há dois meses trabalhando como bolsista no laboratório, e estou por aqui nas tardes de segunda-feira e quarta-feira.

\textit{Você acha que muitas crianças vêm ao laboratório? Quantas, mais ou menos? Em geral, o que elas fazem no computador?}

De tarde são mais as crianças menores, do Ensino Fundamental, e pela manhã vêm mais os alunos do Ensino Médio. Aqui no laboratório os alunos do Fundamental I não podem entrar sem acompanhamento, é necessária a presença de um professor. Geralmente os alunos vêm mais no horário do recreio. De 13h às 15h sempre vem algum aluno para fazer trabalho, mas quando dá 15h30min o pessoal vai para a aula e o laboratório costuma esvaziar.

Ao longo da tarde estimo que em torno de vinte a vinte e cinco pessoas visitem o laboratório. Ocasionalmente recebemos grupos de estudo, de sete a dez alunos, que vêm para fazer trabalho de casa. Normalmente é o pessoal da manhã que fica à tarde fazendo trabalho, ou o pessoal da tarde mesmo.

A maioria dos alunos fica no computador vendo vídeos na internet ou fazendo trabalho, mas alguns querem jogar jogos online; só que esses jogos estão bloqueados, e quando eles percebem o bloqueio, me perguntam “não tem como jogar, tia?”. Eles já sabem o site de jogos, mas ficam tristes porque não conseguem acessá-lo, e, diante da impossibilidade de jogar, se retiram do laboratório.

Tem alunos que ficam vendo vídeos no \textit{YouTube}, mas os vídeos normalmente têm áudio, e os nossos computadores não possuem caixa de som. O laboratório tem um fone de ouvido que é disponibilizado para os alunos, mas é apenas um, e frequentemente algum aluno solicita o seu uso quando ele já está sendo utilizado. Nesse caso, não há o que fazer. Além disso, o fone está quebrado e só um dos lados funciona. O aluno que possui fone de ouvido próprio pode trazer e usar, não há restrição quanto a isso, e alguns alunos ligam o fone do celular no computador. A maioria dos vídeos assistidos pelos alunos no \textit{YouTube} tem conteúdo relacionado a jogos. Além disso, alguns alunos utilizam o \textit{Google Maps} para se distrair.

\textit{Como funciona o processo de supervisão das crianças no laboratório?}

Os avisos fixados nas paredes facilitam a supervisão dos alunos porque são relacionados ao uso do espaço. Se um aluno está utilizando o espaço de modo inadequado, eu chamo a sua atenção mostrando o aviso na parede. A fileira que fica de frente para a sala dos bolsistas é a mais difícil de supervisionar, porque não dá para ver muito bem os alunos, e, devido ao posicionamento dos computadores, eu preciso sair da sala para ver o que os alunos estão fazendo, mas as outras fileiras eu fico olhando da minha sala mesmo. Eles entram no \textit{Instagram}, no \textit{Facebook}.

O processo de supervisão já foi mais difícil. Tínhamos alunos que “matavam aula” para jogar, e por vezes o laboratório ficava muito cheio, acabava dando confusão. Já tentaram instalar um jogo chamado \textit{LoL} em um dos computadores, mas não conseguiram por não terem a senha de administrador.

Hoje em dia o laboratório está mais calmo, as pessoas vêm para imprimir documentos e fazer trabalhos, mas eu fico sempre de olho, porque eventualmente acontece de um computador perder o bloqueio aos jogos; aí a criança que escolheu aquele computador consegue entrar no jogo, mas o colega do computador ao lado não, e essa situação acaba gerando atrito entre as crianças, que a percebem como uma injustiça, e por vezes é necessário intervir. Eu fico observando porque sempre tem um aluno espertinho que tenta burlar a segurança.

\textit{Você ajuda os professores que trazem seus alunos para realizar tarefas no laboratório? Quais as principais demandas que surgem no dia a dia?}

Tem o agendamento da manhã e o da tarde, os professores agendam na planilha que nós mantemos no \textit{Google Sheets}. Eles colocam o dia e o horário no qual vão utilizar o laboratório, e trazem a turma. Nesta ocasião, colocamos um aviso no lado de fora da porta indicando que uma aula está sendo ministrada no laboratório, para evitar que outras pessoas entrem e atrapalhem o bom andamento da aula. Os professores podem utilizar a TV e os computadores, mas o responsável pelo laboratório é o bolsista que se encontra nele. Se, por alguma razão, não houver bolsista naquele momento, o professor pega a chave com os seguranças e se responsabiliza pelo que acontece no laboratório.

Quanto às demandas que os professores trazem para nós, bolsistas, geralmente eles já vêm com um planejamento de aula pronto, e nos informam os recursos que vão precisar utilizar, aí é realizado um direcionamento técnico. Se quiserem usar a televisão, eu aponto qual dos computadores está ligado nela, para que eles possam colocar o vídeo neste computador. Eu também preciso entregar a eles o controle da televisão. É mais o suporte técnico mesmo, os planos de aula já estão prontos quando os professores vêm.

Quanto a outras demandas, poucas vezes eu vi problemas. Por exemplo, antigamente os alunos podiam imprimir sem a nossa supervisão, mas certa feita vários alunos imprimiram documentos simultaneamente, e a impressora travou. O bolsista que estava no laboratório naquela ocasião precisou reiniciá-la e depois organizar os alunos para que as impressões não fossem enviadas ao mesmo tempo. Por conta disso, hoje apenas o computador do bolsista faz impressão, e o aluno precisa solicitá-la ao bolsista. Fora isso, já houve computador precisando ser formatado, mas nunca presenciei nenhuma demanda bizarra, nem nada fora do normal. Nunca tive nenhum aluno rebelde querendo criar confusão, nem nada do tipo. A maioria dos alunos entra e já começa a mexer em tudo no computador. Eles só perguntam se o laboratório está aberto. Quem tem seu fone de ouvido o coloca, faz o que veio fazer, e depois se retira. Normalmente apenas os alunos muito pequenos pedem algum tipo de ajuda.

Além disso, tem a demanda do ar condicionado. A água do condensador pinga dentro do laboratório, e nós colocamos um copo vazio para evitar que a água se espalhe, mas sempre tem alguém que esquece o aparelho de ar condicionado ligado na hora de ir embora, e o copo transborda, fazendo com que o laboratório fique molhado.

\section{Entrevista com o bolsista Pedro}\label{chp:LABEL_CHP_ENT_SEC_PED}

A presente entrevista foi realizada com o bolsista Pedro, que trabalha no LIE, no dia 22 de novembro de 2018, dentro do próprio laboratório.

\textit{Qual o seu curso na UFRJ? Há quanto tempo você está trabalhando como bolsista do laboratório?}

Estou trabalhando aqui no laboratório como bolsista há pouco mais de um ano. Sou aluno da Licenciatura em Educação Física.

\textit{Você acha que muitas crianças vêm ao laboratório? Quantas, mais ou menos?}

Primeiramente, existe uma regra por aqui: crianças do Fundamental I não podem usar os computadores sozinhas, apenas com o acompanhamento de um responsável, que normalmente é o professor, então essas crianças dependem da presença de um professor para poderem vir o laboratório. Dito isso, estimo que, por dia, em média uns dez alunos visitem o laboratório, mas esse número sofre variações ao longo do ano. Agora em novembro, por exemplo, o número de alunos que visita o laboratório diminui muito, porque só estão aqui na escola os alunos que ficaram em recuperação; entretanto, nos meses de agosto e setembro, por exemplo, nós recebemos muitos alunos e esse número é bem mais expressivo.

\textit{Em geral, o que os alunos fazem no computador?}

Embora o laboratório tenha seu uso destinado a servir de apoio para a realização de trabalhos, esta atividade não é a mais comum por aqui. Na verdade, a impressão que tenho é que os trabalhos são realizados aqui apenas em último caso. Normalmente os alunos procuram jogos \textit{on-line}, mas a maior demanda mesmo é por vídeos no \textit{YouTube}. É muito comum que eles vejam vídeos com conteúdo de jogos, inclusive de jogos cuja classificação etária está acima da idade deles. Não sei se os pais ou responsáveis estão cientes disso, ou se permitem esse acesso em casa, e por isso eles reproduzem esse comportamento aqui. É uma possibilidade.

Além disso, já houve acesso a site de conteúdo pornográfico, mas foi uma vez só. O aluno que realizou esse acesso deixou uma imagem com este conteúdo no plano de fundo do computador, e foi um outro aluno que viu e avisou ao bolsista que estava no laboratório na ocasião.

Outro problema pontual foi que certa vez uma professora utilizou o laboratório e, na hora de ir embora, esqueceu uma conta pessoal aberta, aí o aluno que usou aquela mesma máquina depois obteve acesso a tudo, e fez algumas besteiras. Na ocasião isso gerou uma certa confusão pra nós aqui no laboratório, mas foi uma vez só também. Problemas mais graves como esses que eu relatei são bastante raros de acontecer.

\textit{Como funciona o processo de supervisão das crianças no laboratório?}

A principal função do bolsista é tomar conta do laboratório. Nesse contexto, estamos disponíveis para possíveis demandas de algum professor, como realizar uma organização geral na sala ou fazer a instalação de programas que os professores solicitem. Além disso, realizamos a supervisão do laboratório para verificar se o uso que os alunos estão fazendo dele é adequado, se eles precisam de ajuda para obter acesso a algum programa no computador ou a algum conteúdo na internet. No mais, nosso trabalho incluir realizar a impressão de documentos ou trabalhos, conforme for solicitado pelos alunos e professores. O processo de supervisão dos alunos propriamente dito se resume mais a observá-los para verificar se estão fazendo uso adequado dos computadores e do laboratório como um todo, e intervir quando necessário.

\textit{Você ajuda os professores que trazem seus alunos para realizar tarefas no laboratório?}

Os professores que vêm ao laboratório estão sempre precisando de ajuda, e eu ofereço todo o suporte que consigo. As aulas são ministradas no laboratório seguindo um sistema de agendamento: há uma planilha disponibilizada no \textit{Google Sheets} que os professores podem acessar e editar. O professor interessado em dar aula no laboratório procura na planilha um horário que esteja vago e coloca seu nome. A partir daí, neste horário o laboratório estará reservado para ele. Entretanto, recebemos poucos professores… Por mês acontecem em torno de sete aulas, ou seja, menos de duas por semana, se você considerar que o mês tem quatro semanas. Eu queria mais a presença dos professores com suas turmas por aqui. Como agravante, muitos professores vêm apenas para usar a televisão como recurso multimídia, para passar um filme, ou um vídeo; ou seja, além de termos poucas aulas, muitas delas não usam os recursos computacionais do laboratório.

Uma coisa que observo é que há uma certa satisfação dos alunos quando eles vêm com seus professores ao laboratório. Parece que eles ganham o dia, porque se sentem muito bem aqui. A sala tem ar condicionado, o conteúdo ministrado é bem diferente do que é visto no dia-a-dia da sala de aula, etc. Os conteúdos abordados aqui envolvem sempre multimídia, então há muita interação, tanto dos alunos com o computador quanto dos alunos com outros alunos. O professor pede para que todos façam uma mesma tarefa, cada aluno ou dupla de alunos na sua máquina, e isso ajuda o estudante que não sabe usar o computador, porque há a possibilidade de contar com a ajuda dos colegas, e eles se sentem mais seguros com isso. Muitos dos alunos que vêm não conhecem os programas básicos do computador, como o navegador da internet ou o editor de textos, e não é questão de idade, porque isso ocorre também com alunos mais velhos, do Ensino Médio.

Sinto bastante falta de uma aula ou \textit{workshop} voltado diretamente para o laboratório e para o seu uso. Eu participaria dele também, pois é uma boa oportunidade de reciclagem, de se manter atualizado, bem como de conhecer mais programas educacionais, para podermos indicar para os alunos. Seguindo nessa linha, seria positivo também um \textit{workshop} que trabalhasse melhores maneiras de se relacionar com o espaço do laboratório, para conscientizar os alunos nesse sentido.

\textit{Quais as principais demandas que surgem no dia a dia?}

Nós não temos muitos computadores aqui no laboratório, e o nosso espaço não é tão grande, então, quando um grupo grande de alunos é trazido para cá, se faz necessário que os alunos dividam os computadores. As aulas funcionam bem quando eles são divididos em duplas, mas se colocarmos três alunos na mesma máquina a situação já se torna caótica, então existe essa demanda de dividir bem os alunos entre os computadores e supervisionar o uso que eles estão fazendo dos recursos.

Uma demanda enorme que nós recebemos é a de impressão de provas e trabalhos, tanto por parte dos alunos quanto por parte dos professores. Eles sempre me procuram para realizar alguma impressão, essa demanda é muito recorrente; entretanto, ainda assim eu observo que alguns alunos fazem suas pesquisas na internet e anotam o que acham relevante utilizando papel e caneta, então é perceptível que esses alunos ainda não absorveram a cultura do digital, porque não usam um recurso do laboratório, mesmo que estejam dentro dele e este recurso esteja disponível.

Fora isso, as demandas que recebemos vêm mais dos professores. Eles solicitam que programas sejam instalados antes das aulas, para que possam ser usados sem problemas quando os alunos estiverem no laboratório; solicitam que os vídeos a serem exibidos sejam testados antes das aulas, para garantir que não haverá problemas como ausência de áudio ou algum tipo de incompatibilidade. Os professores que vêm ao laboratório são sempre os mesmos, uns quatro ou cinco, então estes docentes já sabem os procedimentos que adotamos, já sabem que as coisas aqui funcionam assim.

\textit{Você acha que o laboratório atrairia mais os alunos se oferecesse mais coisas a eles?}

Com certeza. Se o laboratório oferecesse uma variedade maior de coisas, mais materiais, mais conteúdos interativos, os atrativos para os alunos seriam muito maiores, e provavelmente a quantidade deles dentro do laboratório também. Atualmente a TV gera mais movimento no laboratório, mas mesmo esse recurso poderia ser melhor explorado, para aumentar ainda mais o fluxo de alunos aqui.

Um exemplo de recurso que certamente aumentaria a quantidade de alunos no laboratório é o fone de ouvido. Se um professor quiser realizar no computador uma atividade que precise de som, essa atividade não poderá ser realizada, porque os computadores não possuem caixa de som e o laboratório não possui fones de ouvido, há apenas um, que está com defeito. Da parte dos alunos, essa é uma demanda muito recorrente, pois eles sempre solicitam fones de ouvido e nós nunca temos para oferecer.

\textit{Quais são os principais problemas que você identifica no laboratório?}

O ar condicionado é um grave problema atualmente, pois ele fica pingando dentro do laboratório, em cima da tomada onde o próprio aparelho de ar condicionado está ligado. Como a tomada é externa, nós colocamos um copo apoiado sobre ela para coletar a água que pinga, porém temos que prestar muita atenção no copo, porque ele enche rápido, e se ninguém esvaziá-lo ele transborda e o laboratório fica molhado, então eventualmente é necessário pegar o copo, levar para fora do laboratório e jogar a água no ralo.

Outro grave problema é a questão da manutenção. Atualmente, ela é única e exclusivamente realizada por nós, bolsistas. Sou eu e alguns outros bolsistas que fazemos a manutenção do laboratório inteiro. A impressora, por exemplo, quebrou há algumas semanas, e nenhum dos bolsistas sabe como consertar. Como eu mencionei anteriormente, a demanda por impressão aqui é enorme, e atualmente estamos sem condições de atendê-la; os alunos não podem imprimir seus trabalhos e os professores não podem imprimir suas provas. Certa vez eu escutei que a UFRJ não faz manutenção dos equipamentos eletrônicos, e que há várias impressoras em diversos setores com problemas e defeitos, e nada pode ser feito. Não sei se isso é verdade, mas se for, é um desperdício enorme de dinheiro público, na minha opinião.

Fora isso, são problemas pontuais. Veja a tinta das paredes, por exemplo. Se esse laboratório fosse pintado com uma cor mais chamativa, mais viva, ficaria mais atrativo; entretanto, parece que sempre é difícil pedir coisas simples, como um galão de tinta, pro governo.

Apesar de todos os problemas, eu vou sentir saudades do laboratório. Eu já cumpri os requisitos para me formar, mas estou atrasando propositadamente a abertura do meu processo de colação de grau, objetivando poder ficar aqui por mais tempo, até o final do ano. Muitas escolas não têm os recursos que esse laboratório possui. Eu digo aos alunos que eles têm sorte de ter esse laboratório aqui no CAp.

\textit{Você está se formando em Educação Física. Como docente deste componente curricular, você usaria o laboratório? Como?}

Eu já usei a TV para abordar esportes olímpicos com os alunos. Posso pedir para cada um deles pesquisar sobre esportes olímpicos no computador, ver, anotar coisas diferentes que eles encontrarem. Abordaria outras questões além do esporte em si, talvez usasse para anatomia, para mostrar os músculos do corpo e como eles funcionam. Eu já pensei em trazer um \textit{Kinect} para o laboratório, para os alunos usarem. Olha que interessante, incluir no programa de ensino de atletismo alguns jogos relacionados ao esporte que utilizem o \textit{Kinect}. Se eu tivesse um projetor e um \textit{X-Box}, com certeza já teria feito algo nessa linha.

\textit{Ouvimos da bolsista Ingrid que certa vez tentaram instalar o} LoL \textit{em um dos computadores. Esse é um jogo muito famoso, é o maior expoente do} e-sports\textit{, que é uma área que se integra com a Educação Física, sua área de formação. Considerando que a procura por jogos aqui no laboratório é grande, você o utilizaria para fazer algo na área de} e-sports \textit{com os alunos do CAp?}

Com certeza, fazer algo na área de \textit{e-sports} é uma ideia fantástica! Eu facilmente faria um projeto nesse sentido, pois a adesão dos alunos sem dúvida seria muito forte, o \textit{LoL} e o \textit{e-sports} no geral possuem um apelo muito forte na faixa etária dos nossos alunos. Primeiramente, eu teria que levar esse projeto à DALPE, mas acho que o retorno seria positivo, pois a DALPE costuma gostar de ideias inovadoras, e \textit{e-sports} dentro da escola certamente é muito inovador. O único desafio que eu vislumbro nesse sentido seria mais a organização geral mesmo, que, a princípio, ficaria completamente por minha conta.

\section{Observações e Conclusões}\label{chp:LABEL_CHP_ENT_OBS_CONC}

As entrevistas que realizamos com os diversos atores ligados ao LIE foram fundamentais para termos uma melhor percepção do \textit{modus operandi} do CAp e do laboratório em si. A partir delas foi possível observar e concluir diversas coisas.

O LIE do CAp não possui o seu próprio docente, ator classificado como indispensável por Chagas \cite{art:REF_ART_CHAGAS}; entretanto, esta função é, até certo ponto, desempenhada por bolsistas da própria UFRJ. Portanto, conclui-se que, apesar da ausência do docente do LIE no laboratório do CAp, a situação não é tão precária quanto a que Odorico \cite{art:REF_ART_ODORICO} encontrou nas escolas que visitou na rede estadual de Minas Gerais, pois estas não possuíam nem mesmo bolsistas ou estagiários.

Entretanto, pudemos perceber que a presença dos bolsistas não é suficiente para fomentar a implementação da Informática Educativa, fato evidenciado nas duas entrevistas realizadas com os gestores (DG e DALPE), que demonstraram desejo de que o laboratório fosse mais utilizado para fins pedagógicos, tanto no contexto de cada disciplina isolada quanto no contexto do colégio como um todo.

Fica evidente que os bolsistas não têm possibilidade de desenvolver projetos na área de IE, ainda que estes demonstrem vontade de fazê-lo, vontade esta que se mostra presente, conforme evidenciado pelo bolsista Pedro no final de sua entrevista. Isto se dá pois os bolsistas não possuem vínculo empregatício com o CAp, e não podem ser responsabilizados por projetos ou por eventuais problemas que venham a acontecer no laboratório.

Por conta da ausência do docente do LIE, os bolsistas acabam também assumindo encargos que não deveriam (e talvez nem poderiam) ser seus, tais como a manutenção de hardware e software das máquinas no laboratório, bem como a responsabilidade de decidir se um software deve ou não ser instalado no computador (encargo evidente no relato da visita que realizamos) ou se um site ou software deve ser acessado ou não pelos alunos.

Neste contexto, identificamos uma contradição nas entrevistas. A Profª. Isabel, gestora da DALPE, relatou que o uso de jogos era liberado no laboratório, entretanto, a bolsista Ingrid relatou que os jogos estavam com acesso bloqueado, fato que pôde ser comprovado por nós naquela data. Levantam-se, então, os seguintes questionamentos: a Profª. Isabel está ciente da existência deste bloqueio? Este bloqueio foi autorizado no espaço de tempo entre as entrevistas com a Profª. Isabel (18/10) e a bolsista Ingrid (14/11)? Ou algum bolsista decidiu realizar o bloqueio sem submeter a questão à DALPE, setor responsável pelo laboratório? A dúvida a respeito da sinergia entre os bolsistas e a DALPE reforça ainda mais a importância da presença do docente do LIE.

Outra observação importante é a respeito da nomenclatura do laboratório. A Profª. Isabel se referiu a ele como "LIG", que significa Laboratório de Informática de Graduação, nome que não reflete com exatidão os objetivos pedagógicos do ensino básico. Não temos por objetivo discutir assuntos pouco relevantes, como a nomenclatura adotada; entretanto, este é um indício de que talvez a Informática Educativa não esteja muito bem definida para a gestão do CAp.

No mais, as entrevistas permitiram o conhecimento da dinâmica de funcionamento do laboratório: ele fica aberto aos alunos, que podem entrar e sair livremente se forem do Fundamental II ou do Ensino Médio, e podem usar o computador da maneira que quiserem, respeitadas algumas restrições. Se um professor quiser ministrar uma aula no laboratório, é necessário reservá-lo, e esta reserva é realizada através de uma planilha \textit{on-line} que fica disponível para os professores. O suporte técnico para a aula pode ser solicitado pelo professor e é oferecido pelos bolsistas. Além dos computadores, o laboratório possui um aparelho de televisão.
