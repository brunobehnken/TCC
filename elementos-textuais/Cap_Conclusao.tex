\chapter{Conclusão}\label{chp:LABEL_CHP_CONC}

% \section{Introdução}\label{chp:LABEL_CHP_CONC_SEC_INTRO}

A partir das dificuldades encontradas, que foram relatadas no capítulo anterior, pudemos perceber que o uso do LIE é uma tarefa relativamente complicada e difícil mesmo para os docentes da própria instituição, que, além de não encontrarem suporte extensivo do ponto de vista técnico e pedagógico, ainda precisam lidar com dificuldades que acabam se tornando barreiras à utilização do laboratório, e que atuam, naturalmente, como fator desmotivador a esta utilização.

Percebemos, assim, que fomentar o uso do laboratório por parte dos docentes é um objetivo mais importante do que pensáramos a princípio, e que esse fomento deve começar com a tentativa de amenizar as dificuldades que o laboratório enfrenta e, assim, suavizar as barreiras nas quais estas se constituem.

Refletindo sobre o que aprendemos no referencial teórico e sobre as dificuldades encontradas, chegamos à conclusão de que a maioria delas seria sanada, ou pelo menos amenizada, com a presença do "docente do LIE" no laboratório do CAp.

Neste sentido, optamos por elaborar uma conclusão diferente da que imaginamos anteriormente para este trabalho, atendendo melhor ao objetivo de fomentar o uso do LIE entre os docentes do CAp-UFRJ.

\section{Avaliação das Possibilidades}\label{chp:LABEL_CHP_CONC_SEC_POSS}

Para que alguém assuma a posição de docente do LIE no CAp-UFRJ, vislumbramos duas possibilidades: a de deslocar um funcionário da escola, preferencialmente um professor, para esta função, que seria a opção preferencial, por ser menos burocrática e ter trâmite mais célere; ou a de contratar um professor especialmente para este cargo.

Apresentamos as opções à DALPE em uma reunião para colher \textit{feedback} a respeito das possibilidades de implementação das soluções por parte do CAp. Foi-nos dito que não há possibilidade de deslocar algum funcionário para esta função pois, por força contratual, a carreira dos professores já possui divisões de carga horária, que seriam violadas caso houvesse tal deslocamento.

Entretanto, a DALPE mostrou-se favorável à possibilidade de contratar um novo funcionário para assumir a posição de docente do LIE. Neste contexto, para dar suporte a esta contratação, decidimos por delinear os encargos, a formação necessária e a carga horária relacionados à função de docente do LIE, bem como tecer considerações a respeito de como deve se dar a sinergia entre este e os demais atores.

\section{O Cargo de Docente do LIE}\label{chp:LABEL_CHP_CONC_SEC_DOC}

\subsection{Encargos}\label{chp:LABEL_CHP_CONC_SEC_DOC_SUBSEC_ENC}

A seguir, enumeramos de maneira mais detalhada os encargos do docente do LIE:

\begin{itemize}
\item Zelar pelas condições de uso do laboratório, tanto realizando manutenção preventiva e corretiva de hardware e de software nos computadores, quanto monitorando o uso por parte dos alunos (ex.: estar atento a comportamentos agressivos ou que possam comprometer a infra-estrutura do laboratório). Estima-se que, por semana, o docente deva dedicar em torno de 3 (três) horas para esta atividade.

\item Zelar pelas condições do espaço do laboratório (ar condicionado, iluminação, televisão, infra-estrutura). Gerar relatórios para a DALPE solicitando o que for necessário (ex.: manutenção do ar condicionado). Estima-se que, por semana, o docente deva dedicar em torno de 1 (uma) hora para esta atividade.

\item Oferecer aos professores o suporte técnico necessário em relação à instalação e ao uso dos softwares desejados pelos mesmos. Estima-se que, por semana, o docente deva dedicar em torno de 4 (quatro) horas para esta atividade.

\item Oferecer aos professores o suporte pedagógico necessário para otimizar o uso do laboratório nas suas aulas, desde a etapa do planejamento até a execução (ex.: auxiliar no planejamento das aulas considerando as limitações do laboratório e os objetivos pedagógicos pensados pelo professor, auxiliar o professor na execução das aulas lidando diretamente com os alunos, etc). Estima-se que, por semana, o docente deva dedicar em torno de 4 (quatro) horas para esta atividade.

\item Elaborar propostas pedagógicas de uso do laboratório em projetos multidisciplinares, de modo a envolver a escola como um todo neste uso (ex.: feira de ciências). Estima-se que, por semana, o docente deva dedicar em torno de 4 (quatro) horas para esta atividade.

\item Gerenciar os bolsistas do laboratório. Esta atribuição envolve definir o perfil desejado para o bolsista, realizar o processo seletivo, distribuir tarefas, gerenciar escala de horários, gerenciar frequência de trabalho, avaliar a execução das tarefas, realizar reuniões de acompanhamento, entre outras atividades inerentes à gerência de bolsistas. Estima-se que, por semana, o docente deva dedicar em torno de 4 (quatro) horas para esta atividade.
\end{itemize}

\subsection{Formação Acadêmica}\label{chp:LABEL_CHP_CONC_SEC_DOC_SUBSEC_FORM_ACAD}

Para estar de acordo com o perfil desejado e ser capaz de cumprir com os encargos supracitados, é necessário que o funcionário que irá assumir a função de docente do LIE possua formação acadêmica adequada. Consideramos que o funcionário a ser selecionado atingirá o perfil desejado se possuir, no mínimo, uma das seguintes formações:

\begin{itemize}
\item Curso de graduação em Licenciatura em Informática;

\item Curso de graduação em qualquer Licenciatura ou curso de graduação em Pedagogia ou curso de graduação na área de Computação, acrescido de curso de pós-graduação na área de Informática Educativa.
\end{itemize}

Caso o funcionário, após assumir a função de docente do LIE, observar dificuldades na realização de alguma de suas atribuições, ele deve se sentir livre para realizar cursos de qualificação profissional ou de especialização, ato que deve ser incentivado e facilitado pelo CAp.

\subsection{Carga Horária}\label{chp:LABEL_CHP_CONC_SEC_DOC_SUBSEC_CAR_HOR}

Considerando as estimativas de horas que o docente deve dedicar a cada uma de suas atividades (vide enumeração detalhada dos encargos do docente, acima), recomenda-se que o docente do LIE tenha um regime de 20h semanais, e que alterne os horários nos quais estará presente, de modo que esteja no colégio em um dia na parte da manhã e no dia subsequente na parte da tarde. Os horários nos quais o docente não estiver no laboratório deverão ser preenchidos pelos bolsistas, de modo que estes possuam um horário em que estejam no laboratório sozinhos e um horário em que estejam no laboratório com a presença do docente, para que possa ser realizado o acompanhamento das atividades.

\subsection{Atribuições da DALPE e da DG}\label{chp:LABEL_CHP_CONC_SEC_DOC_SUBSEC_ATRIB}

É preciso ter em mente que, embora o docente do LIE seja soberano no uso do laboratório, ele está sujeito à administração escolar, que, na figura da DALPE e da DG, também possui atribuições, cujo cumprimento se faz necessário para o bom funcionamento do LIE.

Considerando que o Laboratório de Informática Educativa está subordinado à DALPE, o docente do LIE será lotado como funcionário deste setor, tendo como superior imediato o docente que estiver na chefia da DALPE, e tendo como setor superior a DG, conforme o organograma do CAp.

Neste contexto, é atribuição da DALPE observar o cumprimento das atribuições do docente do LIE e avaliá-las quanto ao alcance dos seus objetivos, bem como receber e analisar os relatórios provenientes do docente do LIE, e providenciar o atendimento às demandas identificadas pelo docente. Caso a demanda apresentada não esteja dentro da esfera de competências da DALPE, cumpre a esta atuar como interface entre o docente do LIE e a DG, encaminhando-a para ser solucionada.

Cabe à Direção Geral a solução de problemas estruturais e de outros problemas que venham a ser identificados pelo docente do LIE e encaminhados pela DALPE.

Conforme o exposto, fica evidente que a implementação da Informática Educativa no CAp é atribuição não apenas do docente do LIE, mas também da DALPE e da DG; e que a eficiência dessa implementação é intrinsecamente dependente da sinergia entre estes três atores.

\section{Considerações Finais}\label{chp:LABEL_CHP_CONC_SEC_CONS_FIN}

Embora nosso trabalho não tenha prosseguido da maneira planejada e não tenha chegado às conclusões que imagináramos em princípio, acreditamos que pudemos dar boas contribuições para o CAp-UFRJ.

Foi-nos relatado pela DALPE que já está em estudo a possibilidade de realização de \textit{workshops} para os professores a respeito do uso do laboratório na educação, com propostas pedagógicas que utilizem a informática, e que essa proposta surgiu em reuniões e discussões que foram desdobramentos das nossas visitas ao CAp e das entrevistas que realizamos com os atores envolvidos no laboratório.

Além disso, a demanda pelo docente do LIE nos foi apresentada nas entrevistas aos gestores, antes mesmo de concluirmos que este era o ator que faltava para um melhor funcionamento do laboratório. A conclusão deste trabalho, que apresenta o delineamento dos encargos do docente do LIE e a formação acadêmica necessária para exercer esta função, é uma contribuição importante pois fornece embasamento técnico a um eventual pedido de contratação de funcionário para ocupar esta função; bem como fornece embasamento bibliográfico consagrado, uma vez que a conclusão encontra sustento no referencial teórico apresentado.

\section{Trabalhos Futuros}\label{chp:LABEL_CHP_CONC_SEC_TRAB_FUT}

Se, por um lado, esperamos que a contratação do docente do LIE ocorra, por outro lado essa contratação não elimina a possibilidade de realização de mais trabalhos de naipe semelhante ao nosso no CAp. Existe a possibilidade de ser criado um projeto no qual alunos de computação possam prestar assistência técnica aos professores do CAp na elaboração de propostas pedagógicas que utilizem os recursos computacionais do laboratório.

Este projeto, além de ser uma boa contribuição para o CAp e para os seus professores, servirá ainda como complemento à formação acadêmica dos alunos dos cursos de computação, principalmente dos que se interessam pela área de Informática Educativa e que não encontram muitos ensejos de desenvolver suas aptidões pedagógicas.

A possibilidade de realização de um projeto nesta linha já está sendo analisada pelos professores do Departamento de Ciência da Computação da UFRJ, fruto do nosso contato com o CAp e da nossa parceria com os seus professores.
