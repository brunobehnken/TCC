\chapter{Conclusão}\label{chp:LABEL_CHP_CON}

\section{Considerações Finais}\label{chp:LABEL_CHP_CONC_SEC_CONS_FIN}

Embora nosso trabalho não tenha prosseguido da maneira planejada e não tenha chegado às conclusões que imagináramos em princípio (ou seja, que atendessem aos objetivos da pesquisa), acreditamos que pudemos dar boas contribuições para o CAp-UFRJ.

Foi-nos relatado pela DALPE que já está em estudo a possibilidade de realização de \textit{workshops} para os professores a respeito do uso do laboratório na educação, com propostas pedagógicas que utilizem a informática, e que essa proposta surgiu em reuniões e discussões que foram desdobramentos das nossas visitas ao CAp e das entrevistas que realizamos com os atores envolvidos no laboratório.

Além disso, a demanda pelo docente do LIE nos foi apresentada nas entrevistas aos gestores, antes mesmo de concluirmos que este era o ator que faltava para um melhor funcionamento do laboratório. A conclusão deste trabalho, que apresenta o delineamento dos encargos do docente do LIE e a formação acadêmica necessária para exercer esta função, é uma contribuição importante pois fornece embasamento técnico a um eventual pedido de contratação de funcionário para ocupar esta função; bem como fornece embasamento bibliográfico consagrado, uma vez que o plano de ação encontra sustento no referencial teórico apresentado.

Na bibliografia consultada pudemos observar que as escolas, em geral, não estão bem preparadas e não têm recursos humanos para comportar o uso dos laboratórios, o que torna o funcionamento precário, com pontos frágeis. Em nosso trabalho vivenciamos este aspecto na prática, e confirmamos o que observamos no referencial teórico.

Esta falta de preparo em termos de recursos humanos se deve, em grande parte, a uma característica da gestão de escolas: esta se encontra, muitas vezes, mais preocupada em realizar a compra de equipamentos do que em valorizar o recurso humano, seja disponibilizando-o através de contratações, ou capacitando-o através de treinamentos.

Observamos também que os recursos tecnológicos em geral requerem mais manutenção do que os recursos físicos. Concluímos que o ideal é que esta manutenção seja feita de maneira rotineira, e não apenas conforme for necessário; entretanto, esta segunda maneira de realizar manutenção é a que parece ser mais frequente, conforme observamos tanto no referencial teórico quanto na prática.

\section{Trabalhos Futuros}\label{chp:LABEL_CHP_CONC_SEC_TRAB_FUT}

Se, por um lado, esperamos que a contratação do docente do LIE ocorra, por outro lado essa contratação não elimina a possibilidade de realização de mais trabalhos de naipe semelhante ao nosso no CAp. Existe a possibilidade de ser criado um projeto no qual alunos de computação possam prestar assistência técnica aos professores do CAp na elaboração de propostas pedagógicas que utilizem os recursos computacionais do laboratório.

Este projeto, além de ser uma boa contribuição para o CAp e para os seus professores, servirá ainda como complemento à formação acadêmica dos alunos dos cursos de computação, principalmente dos que se interessam pela área de Informática Educativa e que não encontram muitos ensejos de desenvolver suas aptidões pedagógicas.

A possibilidade de realização de um projeto nesta linha já está sendo analisada pelos professores do Departamento de Ciência da Computação da UFRJ, fruto do nosso contato com o CAp e da nossa parceria com os seus professores.
