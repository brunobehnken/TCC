\chapter{Introdução}\label{chp:LABEL_CHP_INT}

O laboratório de informática do CAp-UFRJ (Colégio de Aplicação da Universidade Federal do Rio de Janeiro) foi reaberto no ano de 2016, e desde então o seu uso livre é permitido aos estudantes do Ensino Médio e Fundamental II, para que estes utilizem os equipamentos para pesquisas, realização de trabalhos, uso de redes sociais e jogos.

Entretanto, percebeu-se que o uso livre do laboratório é majoritariamente para jogos de entretenimento. Considerando que o uso do computador é, por si só, positivo, canalizar tal uso para atividades com foco educacional pode favorecer a melhor compreensão, por parte do aluno, do potencial do computador para o estudo e para a construção do conhecimento (de acordo com a faixa etária) e, consequentemente, gerar melhor usufruto do equipamento disponível no laboratório.

\section{A Parceria com a Profª. Cassandra}\label{chp:LABEL_CHP_INT_SEC_PARC}

Para colocar em prática este objetivo, buscamos a parceria de um docente do CAp, e, em uma reunião do setor multidisciplinar, que cuida das séries iniciais do ensino fundamental, conhecemos a Profª. Cassandra Marina da Silveira Pontes da Silva, que, no ano de 2018, estava responsável por ministrar a disciplina de matemática para uma turma do 5º (quinto) ano fundamental.

Ao conhecer a Profª. Cassandra, apresentamos a ela nossos anseios de canalizar o uso do laboratório para atividades com foco educacional, e perguntamos se haveria possibilidade de estabelecermos uma parceria, de modo que ela ministrasse uma aula no laboratório aos seus alunos contando com o nosso auxílio, ao que ela aquiesceu, visivelmente empolgada com a perspectiva.

Acordamos então que nesta aula seria abordado o conteúdo de frações com uma turma de alunos do 5º (quinto) ano fundamental utilizando o software educacional \textit{JFraction Lab}\footnote{\textit{Software} educacional desenvolvido com a tecnologia \textit{Java} que trabalha o conteúdo de frações utilizando recursos visuais, como gráficos, e possibilitando ao aluno praticar com exercícios de operações básicas, como soma e simplificação de frações.}, e que nós auxiliaríamos a professora nesta aula.

\section{Objeto da Pesquisa}\label{chp:LABEL_CHP_INT_SEC_OBJ}

Este trabalho tem a proposta de abordar, com crianças do 5º (quinto) ano fundamental, alunas do CAp-UFRJ, o conteúdo de frações utilizando os computadores do LIE (Laboratório de Informática Educativa), em parceria com uma docente de Matemática.

\subsection{Objetivo Geral}\label{chp:LABEL_CHP_INT_SEC_OBJ_SUBSEC_OBJ_GER}

Oferecer opções diversificadas de uso do computador para crianças do 5º (quinto) ano fundamental, de forma que elas ampliem seu entendimento sobre o potencial dessa ferramenta e construam conhecimento tanto acerca do conteúdo de frações quanto acerca da informática de um modo geral, respeitadas as limitações da faixa etária.

\subsection{Objetivos Específicos}\label{chp:LABEL_CHP_INT_SEC_OBJ_SUBSEC_OBJ_ESP}

\begin{itemize}
\item Trabalhar o conteúdo de frações utilizando o computador com alunos do 5º (quinto) ano fundamental em parceria com uma docente de Matemática.

\item Desenvolver estratégias para trabalhar o conteúdo de frações com crianças do 5º (quinto) ano fundamental.

\item Fomentar um projeto de extensão no qual os alunos da área de computação possam auxiliar os docentes do CAp-UFRJ na elaboração de propostas didáticas utilizando o computador, e, assim, fomentar o uso do Laboratório de Informática Educativa entre estes docentes.
\end{itemize}

\section{Metodologia}\label{chp:LABEL_CHP_INT_SEC_MET}

A metodologia de pesquisa será pesquisa-ação, ou seja, será desenvolvida uma estratégia que será aplicada e avaliada, e os resultados tem por objetivo o aumento da compreensão do objeto da pesquisa. Pretende-se diagnosticar a abordagem do conteúdo de frações, bem como a utilização do LIE por parte dos professores, por meio de métodos etnográficos (cujo objetivo é o estudo de pessoas e culturas), observação participante, questionários e entrevistas com gestores, professores, bolsistas e alunos; pesquisando soluções possíveis que atendam a demandas específicas do grupo e da comunidade.

\section{Organização do Trabalho}\label{chp:LABEL_CHP_INT_SEC_ORG}

A organização deste trabalho formata-se em cinco capítulos. O primeiro apresenta a introdução ao tema, a proposta da pesquisa e a organização do trabalho.

O segundo capítulo expõe o referencial teórico que fundamenta a inserção das NTIC (Novas Tecnologias de Informação e Comunicação) na escola, tratando de inclusão digital, da informática educativa, do laboratório de informática educativa e do papel dos professores e dos gestores.

O terceiro capítulo apresenta um resumo das entrevistas que realizamos com diversos atores envolvidos na informática educativa no CAp, e tece considerações a respeito delas.

O quarto capítulo expõe o relato da tentativa que fizemos de ministrar uma aula no laboratório em parceria com uma professora, enumera as dificuldades que enfrentamos e articula considerações sobre o que foi exposto.

O quinto capítulo apresenta a conclusão à qual chegamos, bem como a aceitação dela por parte do CAp; tece considerações finais e apresenta possibilidades de trabalhos futuros.
