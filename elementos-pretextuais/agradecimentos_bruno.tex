Em primeiro lugar e acima de tudo, agradeço a Deus por permitir a realização deste trabalho, e ao meu anjo da guarda, que \textbf{sempre} esteve ao meu lado, principalmente nos momentos mais difíceis.

Agradeço à minha avó Nelly pela incansável dedicação e pelos inigualáveis esforços para me dar suporte e zelar por mim, tolerando com paciência e resignação os momentos em que estive ausente ou estressado por estar assoberbado de afazeres.

À minha mãe, Cláudia, pelo amor e carinho dispensados a mim, por todo o apoio incondicional, pela enorme torcida pelo meu sucesso e por todas as orações e promessas que fez por mim.

À minha tia e madrinha, Nelise, pelos conselhos e pelo interesse na minha vida acadêmica e no meu futuro; e ao meu tio, Ricardo, pelo suporte acadêmico e por estar sempre disponível.

À Profª. Juliana, por me motivar em todos os momentos, por orientar a minha iniciação científica e por me convidar para o Inclusive, projeto precursor deste trabalho. À Profª. Carla, pela orientação sincera e descontraída, presente mesmo em meio a tantos afazeres, com reuniões "no encaixe". Ao Prof. Collier, pela impecável orientação acadêmica. Ao Prof. Gabriel, pela monitoria na disciplina de Arquitetura de Computadores, que aguçou meu interesse pela pedagogia.

Ao Colégio de Aplicação da UFRJ, que acolheu o projeto e permitiu o desenvolvimento deste trabalho.

À minha família, de um modo geral, por todo o apoio e compreensão. A todos os meus amigos na faculdade, que fizeram trabalhos comigo e estiveram ao meu lado nos momentos tristes e felizes.