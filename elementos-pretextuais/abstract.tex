The Educational Informatics is an area recognized as necessary for the modernization of the school and its methods, and the computer can act as a catalyst for the teaching and learning process, if well used. However, the mere presence of a laboratory with its computers does not guarantee an efficient implementation of Educational Informatics. It is necessary for teachers to prepare pedagogical proposals for the laboratory, for the laboratory to have its own pedagogical project, for the room to be adequate, for maintenance of computers and other equipment, and for the continuous training of the actors involved in the practice of Educational Informatics; among other impositions.
In this context, this paper proposes to analyze the use and management of the Educational Computer Laboratory of CAp-UFRJ, as well as to promote a better implementation of Educational Informatics in this school, assisting in the elaboration and execution of a class using the laboratory in partnership with a mathematics teacher; besides identifying, discussing and proposing solutions to existing problems.