A Informática Educativa é uma área reconhecidamente necessária para a modernização da escola e de seus métodos, e o computador pode atuar como catalisador do processo de ensino e aprendizado, se bem utilizado. Entretanto, a simples presença de um laboratório com os seus computadores não garante uma implementação eficiente da Informática Educativa. É necessário que os docentes elaborem propostas pedagógicas para o laboratório, que o laboratório possua o seu próprio projeto pedagógico, que a sala seja adequada, que seja realizada manutenção nos computadores e nos demais equipamentos, que seja oferecida formação continuada aos atores envolvidos na prática da Informática Educativa; dentre outras imposições.
Nesse contexto, este trabalho se propõe a analisar o uso e a gestão do Laboratório de Informática Educativa do CAp-UFRJ, bem como a fomentar uma melhor implementação da Informática Educativa neste colégio, auxiliando na elaboração e execução de uma aula utilizando o laboratório em parceria com uma docente de matemática; além de identificar, discutir e propor soluções para problemas existentes.